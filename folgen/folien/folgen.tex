%\documentclass[11pt]{beamer}
\usepackage[ngerman]{babel}
\usepackage[utf8]{inputenc}
\usepackage{amsmath}
\usepackage{amssymb}
\usepackage{listings} 
\usepackage{stmaryrd}
\lstset{language=Python, tabsize=4, showstringspaces=false,basicstyle=\footnotesize,mathescape=true} 
\lstset{literate=%
  {Ö}{{\"O}}1
  {Ä}{{\"A}}1
  {Ü}{{\"U}}1
  {ß}{{\ss}}1
  {ü}{{\"u}}1
  {ä}{{\"a}}1
  {ö}{{\"o}}1
}
\usepackage{mathtools}
\usepackage{ulem}
\usepackage{tikz}

\usetheme{Boadilla}
\mode<presentation>{
\useoutertheme[subsection=false]{miniframes}
\useinnertheme{rectangles}
%\usecolortheme{crane}
}
\parskip 10pt



\begin{document}
\title{Vertiefungskurs Mathematik}   
\author{Folgen} 
\date{}
\frame{\titlepage} 

%---
\begin{frame}[fragile]


\textbf{Definition Folge:} Eine (reelle) Folge ist eine Abbildung $a: \mathbb{N} \rightarrow \mathbb{R}$, also
eine Vorschrift, die jeder natürlichen Zahl $n$ das n-te Folgenglied $a(n) \in \mathbb{R}$ zuordnet. \pause
Wir schreiben $a_n$ für das n-te Folgenglied und $(a_n)$ für die Folge. \\ \pause
\bigskip

Beispiel: $(a_n) = 1, 1, 2, 3, 5, 8, 13, ... $ ist eine Folge mit $a_4 = 3$. \pause

Wir können eine Folge auch ansehen als eine Funktion $f: \mathbb{N} \rightarrow \mathbb{R}$ mit $f(n) = a_n$. \pause

Wir können uns eine Folge vorstellen als eine Folge von Punkten auf der Zahlengeraden. \pause

Manchmal lässt man eine Folge beim Index 0 beginnen.
\end{frame}


%---
\begin{frame}[fragile]
Eine Folge kann durch eine Formel für das n-te Folgenglied gegeben sein. \\ \bigskip \pause

 $a_n = n^2 +1$ \quad beschreibt die Folge  \pause \\ $a_1 = 2, a_2 = 5, a_3 = 10 ...$ \\
\bigskip  \pause

Eine Folge kann rekursiv durch Rückgriff auf frühere Folgenglieder gegeben sein. \\ \bigskip 

$a_n=\begin{cases}
 1,  & n=1, n=2 \\
 a_{n-1}+a_{n-2} & n > 2 
\end{cases} $
 
\end{frame}
 

%---
\begin{frame}[fragile]
Zwei Folgen sind dann gleich, wenn sie mit dem gleichen Index starten und die entsprechenden Folgenglieder alle gleich sind. Dieselbe Folge kann uns auf unterschiedliche Arten gegeben sein.  \pause

$a_n = 2^n$ für $n \ge 0$ \pause

$b_n=\begin{cases}
 1,  & n=0 \\
 2 \cdot b_{n-1} & n > 0 
\end{cases} $  \pause

Die Folgen $(a_n)$ und $(b_n)$ sind gleich.
\end{frame}

%---
\begin{frame}[fragile]
Wir nutzen die Tribonacci Folge $(a_n)$, um daraus eine neue Folge $(b_n)$ zu bauen.  \pause

$a_n=\begin{cases}
 1,  & \text{falls } n=0,1,2 \\
 a_{n-1}+a_{n-2}+a_{n-3} & \text{falls } n > 2
\end{cases} $  \pause

$(a_n) = 1, 1,1,3,5,9,17,31,57,105,...$ \pause

$b_n = \dfrac{a_{n+1}}{a_n}$ \pause

$(b_n) = 1, 1,3,\dfrac{5}{3}, \dfrac{9}{5}, \dfrac{17}{9}, \dfrac{31}{17}, \dfrac{57}{31}, \dfrac{105}{57} ...$ \pause

Die Folge $(b_n)$ mit Dezimalzahlen:
\end{frame}

%---
\begin{frame}[fragile]
\begin{minipage}[t]{4cm}
\begin{lstlisting} 
1.00000000000000
1.00000000000000
3.00000000000000
1.66666666666667
1.80000000000000
1.88888888888889
1.82352941176471
1.83870967741935
1.84210526315789
1.83809523809524
1.83937823834197
1.83943661971831
1.83920367534456
1.83930058284763
1.83929379809869
1.83928131922225
1.83928810384049
1.83928701345944
1.83928642063210
1.83928686638422
\end{lstlisting} 
\end{minipage} 
\begin{minipage}[t]{6cm} \pause
\bigskip
Die Folgenglieder $b_n$ scheinen sich einem Grenzwert $b$ anzunähern.  \\ \pause

Wir schreiben $b = \lim \limits_{n \to \infty} b_n$ \\ \pause

Es kann schwierig sein, den genauen Grenzwert zu berechnen. \pause Für $b_n$ ist es die Zahl: 
\bigskip

$\frac{1}{3}\sqrt[3]{19 + 3\sqrt{33}} - \frac{1}{3}\sqrt[3]{19 - 3\sqrt{33}} + \frac{1}{3}$ 
\bigskip

$\approx 1,8392867552$ 
\end{minipage} 
\end{frame}

%---
\begin{frame}[fragile]
Eine \textbf{arithmetische Folge} ist ein Folge mit einer konstanten Differenz zwischen den Folgengliedern. \\
$5, 12,19,26,33,...$ \pause \quad $a_n = \pause 5 + 7n$. \\ \pause
Allgemeine Form einer arithmetischen Folge: $a_n = a_0 + d \cdot n$. \\ \pause
Jedes Folgenglied ist das arithmetische Mittel seiner Nachbarn. \bigskip \pause

Eine \textbf{geometrische Folge} ist ein Folge mit einem konstanten Quotienten zwischen den Folgengliedern. \\
$3, 6,12,24,48,96..$ \pause \quad $a_n = \pause 3 \cdot 2^n$. \\ \pause
Allgemeine Form einer geometrischen Folge: $a_n = a_0 \cdot q^n$. \\ \pause
Jedes Folgenglied ist das geometrische Mittel seiner Nachbarn. \\ \pause
Das geometrische Mittel zweier Zahlen $a,b$ ist definiert als $\sqrt{a\cdot b}.$
\end{frame}

%---
\begin{frame}[fragile]
Wir untersuchen die Folge $a_n = \frac{6n+2}{3n+3}$ \\ \pause
$a_1 = \pause \frac{8}{6} = \frac{4}{3}$ \\ \pause
$a_{1000} = \pause \frac{6002}{3003} \approx 1.99866799866800$ \\ \pause
$a_{1000000} =  \frac{6000002}{3000003} \approx 1.99999866666800$ \bigskip \pause

Die Folge nähert sich der $2$, wir schreiben: $\lim \limits_{n \to \infty} a_n = 2$. \pause

Damit drücken wir aus: Wir können mit $a_n$ beliebig nahe an die $2$ kommen, wenn wir n nur groß genug wählen. \pause

Für jedes $\epsilon > 0$ gibt es ein $n_0$, so dass $a_n$ nicht mehr als $\epsilon$ von 2 entfernt ist, wenn nur 
$n > n_0$ ist.
\end{frame}

%---
\begin{frame}[fragile]
\textbf{Definition Grenzwert:} Eine Zahl $a \in \mathbb{R}$ heißt Grenzwert der Folge $(a_n)$ wenn gilt:  

$\forall \epsilon > 0 \, \exists n_0 \in \mathbb{N} \, \forall n > n_0 : \vert a_n - a \vert < \epsilon$  \pause

Besitzt eine Folge $(a_n)$ eine Grenzwert $a$ - auch \textit{Limes} genannt - so sagt man, die Folge \textit{konvergiert} 
gegen a und schreibt dafür $\lim \limits_{n \to \infty} a_n = a$ oder $(a_n) \rightarrow a$ für $a \rightarrow \infty$. \pause

Andere Formulierung: $a$ heißt Grenzwert der Folge $(a_n)$, wenn in jeder (noch so kleinen) $\epsilon$-Umgebung von $a$ \textit{fast alle} Elemente der Folge liegen.
\end{frame}

%---
\begin{frame}[fragile]
Gegeben die Folge: $a_n = \frac{n+1}{n+2}$. \quad Behauptung:$\lim \limits_{n \to \infty} a_n = 1$. \pause

Beweis: Sei $\epsilon > 0$. \\ \pause
Wir müssen ein $n_0$ finden, so dass $ \vert a_n - 1 \vert < \epsilon$ für $n > n_0$. \pause

$ \vert a_n - 1  \vert < \epsilon \Leftrightarrow 1 - \frac{n+1}{n+2} < \epsilon \Leftrightarrow (n+2)-(n+1) < \epsilon (n+2)$
$ \Leftrightarrow \frac{1}{\epsilon} < n+2 \Leftrightarrow \frac{1}{\epsilon} - 2 < n .$ \\ \pause
Wähle als $n_0$ eine Zahl mit $n_0 \ge \frac{1}{\epsilon} - 2. \hfill \square$ \pause


Beispiel: für $\epsilon = \frac{1}{100}$ wählen wir $n_0 = 98$. Alle Folgenglieder nach $a_{98}$ haben den Abstand kleiner als $\frac{1}{100}$ zum Grenzwert 1.

\end{frame}

%---
\begin{frame}[fragile]
Folgen sind nützlich für näherungsweise Berechnungen. \pause  Wir betrachten die ersten 7 Elemente der Folge

$x_1= 1, \quad x_{n+1} = \dfrac{1}{x_n} + \dfrac{x_n}{2}$

\begin{lstlisting} 
1.00000000000000
1.50000000000000
1.41666666666667
1.41421568627451
1.41421356237469
1.41421356237310
1.41421356237310
\end{lstlisting} \pause

Die Folge konvergiert gegen $\sqrt{2}$. \pause Wenn man eine gute Näherung für $\sqrt{2}$ benötigt, muss man nur weit genug in der Folge fortschreiten.
\end{frame}

%---

\begin{frame}[fragile]
\textbf{Definition:} Eine Folge $(a_n)$ heißt \textbf{nach oben beschränkt}, wenn es eine Zahl $S \in \mathbb{R}$ gibt, mit 
$a_n  \le S$ für alle $n \in \mathbb{N}$. $S$ heißt dann obere Schranke der Folge. \pause

 Die Folge heißt \textbf{nach unten beschränkt}, wenn es eine Zahl $s \in \mathbb{R}$ gibt, mit 
$s \le a_n$ für alle $n \in \mathbb{N}$. $s$ heißt dann untere Schranke der Folge. \pause

Die Folge heißt \textbf{beschränkt}, wenn sie nach oben und nach unten beschränkt ist. \pause


Beispiele: Die Folge $(a_n)$ mit $a_n = \sin(n)$ \pause ist eine beschränkte Folge. \\
Die Folge $(a_n)$ mit $a_n = n \cdot \sin(\frac{\pi  n}{2})$ \pause ist weder nach oben noch nach unten beschränkt.
\end{frame}

%---
\begin{frame}[fragile]
\textbf{Satz:} Jede konvergente Folge ist beschränkt. \pause

Beweis: Sei $(a_n)$ eine Folge und a ihr Grenzwert. \pause  Wähle $\epsilon = 1$. Dann liegen in der $\epsilon$-Umgebung
$U = (a-1, a+1)$ fast alle Folgenglieder. \pause Die endlich vielen Elemente außerhalb von U haben ein größtes und ein 
kleinstes Element. \pause Das sind die Schranken der Folge. \pause Falls unterhalb oder oberhalb von U keine Elemente vorhanden sind, wählen wir den Rand von U als Schranke. \hfill $\square$
\end{frame}
%---

%---
\begin{frame}[fragile]

\textbf{Grenzwertsätze:}
Für konvergente Folgen $(a_n)$ und $(b_n)$ gilt: 

(G1) Die Summenfolge $(a_n + b_n)$ ist konvergent und ihr Grenzwert ist die Summe der Grenzwerte von
$(a_n)$ und $(b_n)$: \\
$\lim \limits_{n \to \infty} (a_n + b_n)= \lim \limits_{n \to \infty} a_n + \lim \limits_{n \to \infty} b_n$ \pause

(G2) Die Produktfolge $(a_n \cdot b_n)$ ist konvergent und ihr Grenzwert ist das Produkt der Grenzwerte von
$(a_n)$ und $(b_n)$: \\
$\lim \limits_{n \to \infty} (a_n  \cdot b_n)= \lim \limits_{n \to \infty} a_n \cdot \lim \limits_{n \to \infty} b_n$ \pause
 
 (G3) Ist  $\lim \limits_{n \to \infty} b_n \ne 0$, so sind fast alle $b_n \ne 0$, und die (ggf. erst ab einem Index $N>1$ definierte) Quotientenfolge $(\dfrac{a_n}{b_n})$ konvergiert gegen:
 $\lim \limits_{n \to \infty} \frac{a_n}{b_n}=  \frac{\lim \limits_{n \to \infty} a_n }{\lim \limits_{n \to \infty} b_n}$ \pause
 
 Man darf also den Limes in Summe, Produkt und Quotient zweier Folgen `reinziehen',
wenn(!) die Ausgangs-Folgen konvergent sind.
\end{frame}
%---


%---
\begin{frame}[fragile]

Zum Beweis der Grenzwertsätze benötigen wir:

\textbf{Lemma:}
Für zwei reelle Zahlen $x, y \in \mathbb{R}$ gilt die \textit{Dreiecksungleichung}

$\vert x+y \vert \le \vert x \vert + \vert y \vert$ \pause

Beweis: Aus der Definition des Betrags folgt unmittelbar: \\
 $\pm x \le \vert x \vert$ und  $\pm y \le \vert y \vert$ \pause  Also gilt: \\
$x+y \le \vert x \vert + \vert y \vert$ \pause und $-(x+y) = (-x)+(-y) \le  \vert x \vert + \vert y \vert$.  \\ \pause
Insgesamt gilt also: $\vert x + y \vert \le \vert x \vert + \vert y \vert. \hfill \square$. 

\end{frame}
%---

%---
\begin{frame}[fragile]

Beweis von G1:

Sei $\epsilon > 0$ gegeben. \pause Dann gibt es $n_1, n_2 \in \mathbb{N}$ mit $\vert a_n-a \vert < \frac{\epsilon}{2}$
für $n > n_1$ und $\vert b_n-b \vert < \frac{\epsilon}{2}$ für $n > n_2$. \pause Wir setzen $n_0$ als das Maximum von $n_1$
und $n_2$. \pause Dann gilt für alle $n > n_0$: \\
$\vert a_n + b_n - (a + b) \vert = \vert a_n - a + b_n -b \vert \le \vert a_n - a \vert + \vert b_n - b \vert <
 \frac{\epsilon}{2} +  \frac{\epsilon}{2} = \epsilon$ \\ \hfill $\square$ \pause
 
Beweis von G2: \\
 Sei $\epsilon > 0$ gegeben. \pause Es gilt: \\
$\vert a_n b_n - ab \vert = \vert a_n b_n + a_n b - a_n b - ab \vert =  \vert a_n( b_n - b) + (a_n - a)b \vert
\le  \vert a_n( b_n - b)\vert + \vert (a_n - a)b \vert =  \vert a_n\vert \vert ( b_n - b)\vert + \vert (a_n - a)\vert \vert b\vert$. \pause

Da $(a_n)$ konvergiert, gibt es eine Schranke $S$ mit der wir den ersten Summanden abschätzen können. 
 $ \vert a_n\vert \vert ( b_n - b)\vert  \le S  \vert ( b_n - b)\vert$. \pause
 
 Wir wählen $n_1$ und $n_2$ so, dass beide Summanden für größere $n$ kleiner als $\frac{\epsilon}{2}$ sind. \pause
 Für $n > max\{n_1, n_2\}$ gilt dann:
$\vert a_n b_n - ab  \vert  \le S  \vert ( b_n - b)\vert + \vert b\vert\vert (a_n - a)\vert \le \frac{\epsilon}{2} + \frac{\epsilon}{2} = \epsilon$. \hfill $\square$ 

\end{frame}
%---

\begin{frame}[fragile]
\textbf{Definition:} Eine Folge $(a_n)$ heißt \textbf{monoton wachsend}, wenn $a_{n+1} \ge a_n$ für
alle $n\in \mathbb{N}$ gilt. Gilt sogar $>$ anstelle von $\ge$, so heißt die Folge \textbf{streng monoton
wachsend}. \pause

Entsprechend ist \textbf{(streng) monoton fallend} definiert. \pause

Eine Folge heißt \textbf{(streng) monoton}, wenn sie (streng) monoton wachsend oder
(streng) monoton fallend ist. \pause

Beispiele: Die Folge $(a_n)$ mit $a_n = n^2$ \pause ist streng monton wachsend. \pause

Die Folge $(b_n)$ mit $b_n = 1,1,2,2,3,3,4,4,....$ \pause ist monton wachsend, aber nicht streng monoton wachsend. \pause

Andere Formulierung: Eine Folge ist monoton, wenn alle Folgenglieder in dieselbe Richtung gehen.
\end{frame}

\begin{frame}[fragile]
\textbf{Satz (Montoniekriterium):} Jede beschränkte monotone Folge konvergiert. \pause

Statt eines formalen Beweises machen wir uns den Inhalt geometrisch plausibel: \pause Da die Folge eine obere Schranke hat, hat sie auch eine kleinste obere Schranke. Das ist dann der Grenzwert. \pause

Beispiel: Die Folge $(a_n)$ sei gegeben durch $a_1 = 1$ und $a_{n+1} = \sqrt{a_n+2}$. Mit vollständiger Induktion wir zeigen wir, dass $(a_n)$ streng monoton wächst und beschränkt ist: $0 \le a_n \le 2$ für alle $n \in \mathbb{N}$. \pause
Also hat $(a_n)$ einen Grenzwert. \pause Der Grenzwert erfüllt die Gleichung $x = \sqrt{x+2}$,  daraus berechnen wir den Grenzwert $a = 2$. 

\end{frame}

\begin{frame}[fragile]
\textbf{Exkurs:} Gibt es eine Folge, die jede ganze Zahl enthält? \pause

$(a_n): 0,-1,1,-2,2,-3,3,-4,4,..... $ \pause

$a_n=\begin{cases}
 -\frac{n+1}{2},  & \text{falls n ungerade}\\
\frac{n}{2} & \text{falls n gerade }
\end{cases} $  \pause

Zwei endliche Mengen haben gleich viele Elemente, wenn man eine eindeutige Zuordnung zwischen den Elementen der beiden Mengen herstellen kann. \pause Auf unendliche Mengen übertragen zeigt die Folge: Es gibt genauso viele natürliche Zahlen wie ganze Zahlen. \pause

Etwas Unendliches wird nicht notwendig kleiner, wenn man etwas wegnimmt.

\end{frame}


\begin{frame}[fragile]
\textbf{Exkurs:} Gibt es eine Folge, die jede reelle Zahl enthält? \pause 

Einfachere Version: Gibt es eine Folge, die jede reelle Zahl zwischen 0 und 1 enthält? \pause

Annahme: Es gibt Folge $(a_n)$, in der alle reellen Zahlen zwischen 0 und 1 vorkommen. \pause Jedes $a_n$ hat eine Dezimalentwicklung. \pause Sei $D$ die $n$-te Dezimalstelle von $a_n$. \pause Wir konstruieren eine Zahl $x$ mit

Die $n$-te Dezimalstelle von $x = $ 
$\begin{cases}
D+1,  & \text{ falls } D \le 7,\\
D-1  & \text{ falls } D=8 \text{ oder } D=9 
\end{cases} $   \pause

Dann ist $x$ eine reelle Zahl zwischen 0 und 1 \pause, kann aber kein Element der Folge $(a_n)$ sein, da es sich von jedem $a_n$ in mindestens einer Dezimalstelle unterscheidet. \hfill $\square$

\end{frame}


\end{document}