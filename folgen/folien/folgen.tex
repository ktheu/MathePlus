\documentclass[11pt]{beamer}
\usepackage[ngerman]{babel}
\usepackage[utf8]{inputenc}
\usepackage{amsmath}
\usepackage{amssymb}
\usepackage{listings} 
\usepackage{stmaryrd}
\lstset{language=Python, tabsize=4, showstringspaces=false,basicstyle=\footnotesize,mathescape=true} 
\lstset{literate=%
  {Ö}{{\"O}}1
  {Ä}{{\"A}}1
  {Ü}{{\"U}}1
  {ß}{{\ss}}1
  {ü}{{\"u}}1
  {ä}{{\"a}}1
  {ö}{{\"o}}1
}
\usepackage{mathtools}
\usepackage{ulem}
\usepackage{tikz}

\usetheme{Boadilla}
\mode<presentation>{
\useoutertheme[subsection=false]{miniframes}
\useinnertheme{rectangles}
%\usecolortheme{crane}
}
\parskip 10pt



\begin{document}
\title{Vertiefungskurs Mathematik}   
\author{Folgen} 
\date{}
\frame{\titlepage} 

%---
\begin{frame}[fragile]


\textbf{Definition Folge:} Eine (reelle) Folge ist eine Abbildung $a: \mathbb{N} \rightarrow \mathbb{R}$, also
eine Vorschrift, die jeder natürlichen Zahl $n$ das n-te Folgenglied $a(n) \in \mathbb{R}$ zuordnet. \pause
Wir schreiben $a_n$ für das n-te Folgenglied und $(a_n)$ für die Folge. \\ \pause
\bigskip

Beispiel: $(a_n) = 1, 1, 2, 3, 5, 8, 13, ... $ ist eine Folge mit $a_4 = 3$. \pause

Wir können eine Folge auch ansehen als eine Funktion $f: \mathbb{N} \rightarrow \mathbb{R}$ mit $f(n) = a_n$. \pause

Wir können uns eine Folge vorstellen als eine Folge von Punkten auf der Zahlengeraden. \pause

Manchmal lässt man eine Folge auch beim Index 0 beginnen.
\end{frame}


%---
\begin{frame}[fragile]
Manchmal ist uns eine Folge durch eine algebraische Vorschrift gegeben. \\ \bigskip \pause

 $a_n = n^2 +1$ \quad beschreibt eine Folge mit den Folgengliedern  \pause \\ $a_1 = 2, a_2 = 5, a_3 = 10 ...$ \\
\bigskip  \pause
Manchmal ist es schwierig, eine Formel für das n-te Folgenglied zu finden. Eine Folge kann auch rekursiv beschrieben werden: \pause \bigskip 

$a_n=\begin{cases}
 1,  & n=1, n=2 \\
 a_{n-1}+a_{n-2} & n > 2 
\end{cases} $
 
Der Wert eines Folgenglieds wird durch Rückgriff auf frühere Folgenglieder festgelegt. 
\end{frame}
 

%---
\begin{frame}[fragile]
Zwei Folgen sind dann gleich, wenn sie mit dem gleichen Index starten und die entsprechenden Folgenglieder alle gleich sind. Dieselbe Folge kann uns auf unterschiedliche Arten gegeben sein.  \pause

$a_n = 2^n$ für $n \ge 0$ \pause

$b_n=\begin{cases}
 1,  & n=0 \\
 2 \cdot b_{n-1} & n > 0 
\end{cases} $  \pause

Die Folgen $(a_n)$ und $(b_n)$ sind gleich.
\end{frame}

%---
\begin{frame}[fragile]
Wir nutzen die Tribonacci Folge $(a_n)$, um daraus eine neue Folge $(b_n)$ zu bauen.  \pause

$a_n=\begin{cases}
 1,  & \text{falls } n=0,1,2 \\
 2 \cdot a_{n-1}+a_{n-2}+a_{n-3} & \text{falls } n > 2
\end{cases} $  \pause

$(a_n) = 1, 1,1,3,5,9,17,31,57,105,...$

$b_n = \dfrac{a_{n+1}}{a_n}$

$(b_n) = 1, 1,3,\dfrac{5}{3}, \dfrac{9}{5}, \dfrac{17}{9}, \dfrac{31}{17}, \dfrac{57}{31}, \dfrac{105}{57} ...$

Die Folge $(b_n)$ mit Dezimalzahlen:
\end{frame}

%---
\begin{frame}[fragile]
\begin{minipage}[t]{4cm}
\begin{lstlisting} 
1.00000000000000
1.00000000000000
3.00000000000000
1.66666666666667
1.80000000000000
1.88888888888889
1.82352941176471
1.83870967741935
1.84210526315789
1.83809523809524
1.83937823834197
1.83943661971831
1.83920367534456
1.83930058284763
1.83929379809869
1.83928131922225
1.83928810384049
1.83928701345944
1.83928642063210
1.83928686638422
\end{lstlisting} 
\end{minipage} 
\begin{minipage}[t]{6cm}
\bigskip
Die Folgenglieder $b_n$ scheinen sich einem Grenzwert $b$ anzunähern.  \\

Wir schreiben $b = \lim \limits_{n \to \infty} b_n$ \\

Es kann schwierig sein, den genauen Grenzwert zu berechnen. Für $b_n$ ist es die Zahl: 
\bigskip

$\frac{1}{3}\sqrt[3]{19 + 3\sqrt{33}} - \frac{1}{3}\sqrt[3]{19 - 3\sqrt{33}} + \frac{1}{3}$ 
\bigskip

$\approx 1,8392867552$ 
\end{minipage} 
\end{frame}

%---
\begin{frame}[fragile]
Eine \textbf{arithmetische Folge} ist ein Folge mit einer konstanten Differenz zwischen den Folgengliedern. \\
$5, 12,19,26,33,...$ \quad $a_n = 5 + 7n$. \\
Allgemeine Form einer arithmetischen Folge: $a_n = a_0 + d \cdot n$. \\
Jedes Folgenglied ist das arithmetische Mittel seiner Nachbarn. \bigskip

Eine \textbf{geometrische Folge} ist ein Folge mit einen konstanten Quotienten zwischen den Folgengliedern. \\
$3, 6,12,24,48,96..$ \quad $a_n = 3 \cdot 2^n$. \\
Allgemeine Form einer geometrischen Folge: $a_n = a_0 \cdot q^n$. \\
Jedes Folgenglied ist das geometrische Mittel seiner Nachbarn. \\
Das geometrische Mittel zweier Zahlen $a,b$ ist definiert als $\sqrt{a\cdot b}.$
\end{frame}

%---
\begin{frame}[fragile]
Wir untersuchen die Folge $a_n = \frac{6n+2}{3n+3}$ \\
$a_1 = \frac{8}{6} = \frac{4}{3}$ \\
$a_{1000} = \frac{6002}{3003} \approx 1.99866799866800$ \\
$a_{1000000} = \frac{6000002}{3000003} \approx 1.99999866666800$ \bigskip

Die Folge nähert sich der $2$, wir schreiben: $\lim \limits_{n \to \infty} a_n = 2$.

Damit drücken wir aus: Wir können mit $a_n$ beliebig nahe an die $2$ kommen, wenn wir n nur groß genug wählen.

Für jedes $\epsilon > 0$ gibt es ein $n_0$, so dass $a_n$ nicht mehr als $\epsilon$ von 2 entfernt ist, wenn nur 
$n > n_0$ ist.
\end{frame}

%---
\begin{frame}[fragile]
\textbf{Definition Grenzwert:} Eine Zahl $a \in \mathbb{R}$ heißt Grenzwert der Folge $(a_n)$ wenn gilt:  

$\forall \epsilon > 0 \, \exists n_0 \in \mathbb{N} \, \forall n > n_0 : \vert a_n - a \vert < \epsilon$  

Besitzt eine Folge $(a_n)$ eine Grenzwert $a$ - auch \textit{Limes} genannt - so sagt man, die Folge \textit{konvergiert} 
gegen a und schreibt dafür $\lim \limits_{n \to \infty} a_n = a$ oder $(a_n) \rightarrow a$ für $a \rightarrow \infty$.

Andere Formulierung: $a$ heißt Grenzwert der Folge $(a_n)$, wenn in jeder (noch so kleinen) $\epsilon$-Umgebung von $a$ \textit{fast alle} Elemente der Folge liegen.
\end{frame}

%---
\begin{frame}[fragile]
Beispiel: $a_n = \frac{n+1}{n+2}$. \quad Behauptung:$\lim \limits_{n \to \infty} a_n = 1$. 

Beweis: Sei $\epsilon > 0$. \\
Wir müssen ein $n_0$ finden, so dass $ \vert a_n - 1 \vert < \epsilon$ für $n > n_0$.

$ \vert a_n - 1  \vert < \epsilon \Leftrightarrow 1 - \frac{n+1}{n+2} < \epsilon \Leftrightarrow (n+2)-(n+1) < \epsilon (n+2)$
$ \Leftrightarrow \frac{1}{\epsilon} < n+2 \Leftrightarrow \frac{1}{\epsilon} - 2 < n .$ \\
Wähle als $n_0$ eine Zahl mit $n_0 \ge \frac{1}{\epsilon} - 2. \hfill \square$


Beispiel: für $\epsilon = \frac{1}{100}$ wählen wir $n_0 = 98$. Alle Folgenglieder nach $a_{98}$ haben den Abstand kleiner als $\frac{1}{100}$ zum Grenzwert 1.

\end{frame}







\end{document}