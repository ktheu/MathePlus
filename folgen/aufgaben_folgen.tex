\documentclass[landscape,twocolumn,a4paper]{article}
\usepackage[utf8x]{inputenc}
\usepackage[ngerman]{babel}
\usepackage{listings}
\usepackage{babel}

\usepackage[T1]{fontenc}
\usepackage{booktabs} % schöne Tabellen
\usepackage{graphicx}
\usepackage{csquotes} % Anführungszeichen
\usepackage{paralist} % kompakte Aufzählungen
\usepackage{amsmath,textcomp,tikz} %diverses
\usepackage{eso-pic} % Bilder im Hintergrund
\usepackage{mdframed} % Boxen
\usepackage{multirow}
\usepackage{amssymb}

\usepackage{mathtools}
\usepackage[top=20mm,left=10mm,right=10mm,bottom=10mm]{geometry}
\usepackage{fancyhdr}
\pagestyle{fancy}
\fancyhead[L]{Aufgaben der Zertifikatsklausuren}
\fancyhead[R]{\thepage}
\fancyfoot{}

\lstset{language=Python, tabsize=4, basicstyle=\footnotesize, showstringspaces=false, mathescape=true}
\lstset{literate=%
  {Ö}{{\"O}}1
  {Ä}{{\"A}}1
  {Ü}{{\"U}}1
  {ß}{{\ss}}1
  {ü}{{\"u}}1
  {ä}{{\"a}}1
  {ö}{{\"o}}1
}
\begin{document}

\parindent 0mm


\textbf{Folgen, Grenzwerte}
\bigskip 

\textbf{A2021} \\
Es sei $(a_n)$ eine Zahlenfolge. \\
a. Geben Sie die Definition dafür an, dass $\lim \limits_{n \to \infty} a_n = a$ gilt.\\
b. Eine Folge heißt Nullfolge, wenn sie gegen Null konvergiert. Beweisen Sie durch Anwendung
der Definition aus a.,  dass die Folge $(b_n)$ mit $b_n = \dfrac{n}{n^2+1}$ eine Nullfolge ist. \\
c. Beweisen Sie ohne die Verwendung von Grenzwertsätzen: Sind $(a_n)$ und $(b_n)$ Nullfolgen, dann
ist auch die Summenfolge $(a_n + b_n)$ eine Nullfolge. \\
d. Geben Sie Beispielfolgen $a_n$, $b_n$ an, die keine Nullfolgen sind, und deren Summenfolge eine
Nullfolge bildet.
\bigskip 

\textbf{A2020} \\
Gegeben seien reelle Folgen $(a_n)$, $(b_n)$ und reelle Zahlen $a, b$.\\
a. Geben Sie die Definition dafür an, dass $(a_n)$ gegen $a$ konvergiert. \\
b. Gegeben ist der Satz: Konvergieren die Folgen  $(a_n)$ und  $(b_n)$, so konvergiert
auch die Summenfolge $(a_n + b_n)$ \\
b1. Wie lautet der Grenzwert der Folge $(a_n + b_n)$ wenn  $(a_n)$ gegen $a$ und $(b_n)$ gegen $b$
konvergiert? \\
b2. Beweisen Sie den Satz. \\
b3. Formulieren Sie die Umkehrung des Satzes und zeigen Sie, dass diese falsch ist. \\
c. Bestimmen Sie den Grenzwert der Folge $(a_n)$  mit \\
$a_n = \dfrac{n^2 \cdot 2^n + 4^n + (-1)^n}{3^n - 4^n}$
 
\bigskip 

\textbf{A2019} \\
a. Gegeben sind eine reelle Folge $(a_n)$ und eine Zahl $a \in \mathbb{R}$ \\
a1. Geben Sie die Definition der Konvergenz $a = \lim \limits_{n \to \infty} a_n$. \\
a2. Beweisen Sie den folgenden Satz: Ist die Folge $(a_n)$ konvergent, so ist sie beschränkt. \\
a3. Bilden Sie die Umkehrung des Satzes aus a2 und zeigen Sie, dass die Aussage falsch ist. \\
b. Bestimmen Sie den Grenzwert der Folge: $(b_n)$ mit $b_n = \sqrt{n^2+2n} - \sqrt{n^2-n}$ für alle $n \in
\mathbb{N}$
\bigskip 

\textbf{A2018} \\
a. Gegeben Sie in jeder Teilaufgabe ein Beispiel an für Folgen, die die angegebenen Aussagen erfüllen: \\
a1. $(a_n)$ ist konvergent und $(b_n)$ ist divergent und $(a_n * b_n)$ ist divergent \\
a2. $(a_n)$ ist konvergent und $(b_n)$ ist divergent und $(a_n * b_n)$ ist konvergent \\
a3. $(a_n)$ ist divergent und $(b_n)$ ist divergent und $(a_n * b_n)$ ist divergent \\
a4. $(a_n)$ ist divergent und $(b_n)$ ist divergent und $(a_n * b_n)$ ist konvergent \\

b. Es seinen $(a_n),(b_n)$ konvergente reelle Folgen mit $a = \lim \limits_{n \to \infty} a_n$ und
$b = \lim \limits_{n \to \infty} b_n$. Was kann man über die Folge $(a_n * b_n)$ aussagen? (Ohne Beweis!) 

c. Es seinen $(a_n)$ eine gegen $a$ konvergente Folge. Beweisen Sie durch Induktion bezüglich $m$, dass für 
alle $m \in \mathbb{N}$ gilt: Die Folgen $(a_n^m)$ konvergiert gegen $a^m$. \textit{Hinweis:} Verwenden Sie die Aussage aus Teil b.
\bigskip 

\newpage

\textbf{A2017} \\
Mit $(a_n)$ wird eine Folge bezeichnet, die die Folgenglieder $a_n, (n \in \mathbb{N})$ besitzt. \\
a. Es sei $(a_n)$ eine reelle Folge und a eine reelle Zahl. Geben Sie die Definition der Konvergenz von $(a_n)$ geben $a$ an. \\
b. Beweisen Sie, dass  $ \lim \limits_{n \to \infty} \dfrac{1}{2^n} = 0$. Weisen Sie dazu nach, dass die Definition der Konvergenz erfüllt ist. \\
c. Sei $(b_n)$ eine Folge mit $\left|b_n\right| \le \dfrac{1}{2^n}$ für $n \in \mathbb{N}$. Zeigen Sie,
dass $(b_n)$ geben $0$ konvergiert. Weisen Sie dazu nach, dass die Konvergenzdefinition erfüllt ist. \\
d. Sei $(c_n)$ eine Folge mit  $\left|c_n\right| \le \dfrac{1}{2}$ für $n \in \mathbb{N}$. Sie weiter die 
Folge $(d_n)$ definiert durch $d_1 = \dfrac{1}{2}, d_{n+1} = c_n \cdot d_n$ für $n \in \mathbb{N}$.
Beweisen Sie, dass die Folgen $(d_n)$ gegen Null konvergiert. 

\textit{Hinweis:} Sie dürfen in jedem Aufgabenteil die Resultate der davorliegenden Aufgabenteile verwenden, auch wenn Sie diese nicht bewiesen haben.
\bigskip 

\textbf{A2016} \\
Gegeben sei eine reelle Folge $(a_n)$ und eine reelle Zahl $a$. \\
a. Geben Sie die Definition dafür an, dass die Folge $(a_n)$ gegen a konvergiert, also $\lim \limits_{n \to \infty} a_n = a$ \\
b.Weise Sie nach, dass  $ \lim \limits_{n \to \infty} \dfrac{1}{n} = 0$ gilt. \\
c. Es seien $(a_n), (b_n)$ Folgen und es gelte $a_n \le b_n \le a_n + \dfrac{1}{n}$ für $n \in \mathbb{N}$.
Beweisen Sie: Ist $(a_n)$ konvergent gegen $a$, dann konvergiert auch $(b_n)$ gegen $a$. \\
d. Bestimmen Sie durch Anwendung der Sätze über konvergente Folgen unter Zuhilfenahme von Teil c) den Grenzwert der Folge $(b_n)$ mit 
$$b_n := \dfrac{n^4-n^2+5}{(n+3)^2 \cdot (2n-1)^2} + \dfrac{1+(-1)^n}{2n} \cdot  \sin^2(n)$$
\bigskip

\textbf{Orientierungsaufgaben} \\

\textbf{A1} \\
a. Untersuchen Sie die nachstehend gegebene Folge auf Konvergenz und bestimmen Sie gegebenenfalls
ihren Grenzwert durch Anwendung der Sätze über konvergente Folgen. 

$a_n = \dfrac{(n-1)(n+1)}{2n-2} \cdot \dfrac{2n^3+3n^2+5}{3n^4+2n^3+n^2+2}$ 

b. Gegeben sei eine konvergente Folge $a_n$ mit $a_n \in \mathbb{R}, a_n \ge 0$. Zeigen Sie, dass
für den Grenzwert $a = \lim \limits_{n \to \infty} a_n$ die Ungleichung $a \ge 0$ gilt. \textit{Hinweis:} Sie
können beispielsweise die Annahme $a < 0$ zum Widerspruch führen.
\bigskip

\newpage

\textbf{A2} \\
Gegeben ist die Folge $a_n = \dfrac{n^2}{2n^2+5}$. \\
a) Bestimmen Sie den Grenzwert. \\
b) Geben Sie die Definition der Aussage \glqq $a_n$ konvergiert gegen a\grqq\,an. \\
c) Beweisen Sie für die oben angegebene Folge $a_n$ und den von Ihnen gefundenen Grenzwert $a$, dass die
Definition von \glqq $a_n$ konvergiert gegen a\grqq\ erfüllt ist.\\
\bigskip

\textbf{A3} \\
Untersuchen Sie die nachstehenden Folgen auf Konvergenz und bestimmen Sie gegebenenfalls ihre Grenzwerte. \\
a) $a_n = \dfrac{2n+1}{n+3} + \dfrac{4n^2+3n+1}{(2n+1)^2} $\\
b) $a_n = (-1)^n \cdot \dfrac{n+1}{n^2-2} \quad (n \ge 2)$ \\
c) $a_n = \dfrac{1+2+...+n}{n^2} $ \\
d) $a_n = (-1)^n \cdot \dfrac{n^2-1}{n^2+1}$ \\
e) $a_n = \dfrac{2^n-3^n}{2^n+3^n}$

\textbf{A4} \\
Berechnen Sie folgende Grenzwerte \\
a) $\lim \limits_{n \to \infty} (\sqrt{3+2n}-\sqrt{2n})$ \\
b) $\lim \limits_{n \to \infty} (\sqrt{n^2+4n}-\sqrt{n^2+n})$ \\


\end{document}
