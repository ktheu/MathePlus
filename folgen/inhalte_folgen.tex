\documentclass[a4paper]{article}
\usepackage[utf8x]{inputenc}
\usepackage[ngerman]{babel}
\usepackage{listings}
\usepackage{babel}

\usepackage[T1]{fontenc}
\usepackage{booktabs} % schöne Tabellen
\usepackage{graphicx}
\usepackage{csquotes} % Anführungszeichen
\usepackage{paralist} % kompakte Aufzählungen
\usepackage{amsmath,textcomp,tikz} %diverses
\usepackage{eso-pic} % Bilder im Hintergrund
\usepackage{mdframed} % Boxen
\usepackage{multirow}
\usepackage{amssymb}

\usepackage{mathtools}
\usepackage[top=20mm,left=10mm,right=10mm,bottom=10mm]{geometry}
\usepackage{fancyhdr}
\pagestyle{fancy}
\fancyhead[L]{Folgen, Grenzwerte}
\fancyhead[R]{\thepage}
\fancyfoot{}

\lstset{language=Python, tabsize=4, basicstyle=\footnotesize, showstringspaces=false, mathescape=true}
\lstset{literate=%
  {Ö}{{\"O}}1
  {Ä}{{\"A}}1
  {Ü}{{\"U}}1
  {ß}{{\ss}}1
  {ü}{{\"u}}1
  {ä}{{\"a}}1
  {ö}{{\"o}}1
}
\begin{document}

\parindent 0mm


\textbf{Definition Folge:} Eine (reelle) Folge ist eine Abbildung $a: \mathbb{N} \rightarrow \mathbb{R}$, also
eine Vorschrift, die jeder natürlichen Zahl $n$ das n-te Folgenglied $a(n) \in \mathbb{R}$ zuordnet.
Wir schreiben $a_n$ für das n-te Folgenglied und $(a_n)$ für die Folge.
\bigskip

\textbf{Definition Grenzwert:} Eine Zahl $a \in \mathbb{R}$ heißt Grenzwert der Folge $(a_n)$ wenn gilt: \\
$\forall \epsilon > 0 \, \exists n_0 \in \mathbb{N} \, \forall n > n_0 : \vert a_n - a \vert < \epsilon$ \\

Besitzt eine Folge $(a_n)$ eine Grenzwert $a$ - auch \textit{Limes} genannt - so sagt man, die Folge \textit{konvergiert} 
gegen a und schreibt dafür $\lim \limits_{n \to \infty} a_n = a$ oder $(a_n) \rightarrow a$ für $a \rightarrow \infty$.

Andere Formulierung: $a$ heißt Grenzwert der Folge $(a_n)$, wenn in jeder (noch so kleinen) $\epsilon$-Umgebung von $a$ \textit{fast alle} Elemente der Folge liegen.

\bigskip

\textbf{Definition Beschränktheit:} Eine Folge $(a_n)$ heißt beschränkt, wenn es eine Zahl $S \in \mathbb{R}$ gibt, mit 
$\vert a_n \vert \le S$ für alle $n \in \mathbb{N}$.
\bigskip

\textbf{Satz:} Jede konvergente Folge ist beschränkt.
\bigskip

\textbf{Lemma (Dreiecksungleichung):} Für $x,y \in \mathbb{R}$ gilt: $\vert x+y \vert 
\le \vert x \vert + \vert y \vert$
\bigskip

\textbf{Satz:} Falls  $\lim \limits_{n \to \infty} a_n = a$ und $\lim \limits_{n \to \infty} b_n = b$, so gilt:\\
(1)  $\lim \limits_{n \to \infty} (a_n + b_n)= a + b$ \\
(2)  $\lim \limits_{n \to \infty} (a_n \cdot b_n)= a \cdot b$ \\
(3)  $\lim \limits_{n \to \infty} \dfrac{a_n}{b_n}= \dfrac{a}{b}$, falls $b, b_n\neq 0$
\bigskip







\end{document}
