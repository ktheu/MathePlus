\documentclass[a4paper]{article}
\usepackage[utf8x]{inputenc}
\usepackage[ngerman]{babel}
\usepackage{listings}
\usepackage{babel}

\usepackage[T1]{fontenc}
\usepackage{booktabs} % schöne Tabellen
\usepackage{graphicx}
\usepackage{csquotes} % Anführungszeichen
\usepackage{paralist} % kompakte Aufzählungen
\usepackage{amsmath,textcomp,tikz} %diverses
\usepackage{eso-pic} % Bilder im Hintergrund
\usepackage{mdframed} % Boxen
\usepackage{multirow}
\usepackage{amssymb}

\usepackage{mathtools}
\usepackage[top=20mm,left=10mm,right=10mm,bottom=10mm]{geometry}
\usepackage{fancyhdr}
\pagestyle{fancy}
\fancyhead[L]{Zahlentheorie}
\fancyhead[R]{\thepage}
\fancyfoot{}

\lstset{language=Python, tabsize=4, basicstyle=\footnotesize, showstringspaces=false, mathescape=true}
\lstset{literate=%
  {Ö}{{\"O}}1
  {Ä}{{\"A}}1
  {Ü}{{\"U}}1
  {ß}{{\ss}}1
  {ü}{{\"u}}1
  {ä}{{\"a}}1
  {ö}{{\"o}}1
}
\parindent 0mm

\newcommand{\ggT}{\operatorname{ggT}}
\newcommand{\Mod}[3]{#1\equiv#2\mod#3}
\begin{document}

\textbf{Zahlentheorie} ist die Mathematik der ganzen Zahlen.
\bigskip

\textbf{Diophantische Gleichungen} \\
Eine Gleichung der Form $a x + b y = c$ mit $a, b,c \in \mathbb{N}$ und $x,y \in \mathbb{Z}$ heißt
\textit{diophantische Gleichung}. \\
In der Regel sind $a,b,c$ gegeben und $x,y$ gesucht. \\
 Wir schreiben 
Lösungen als Zahlenpaar $(x/y)$.
\bigskip

\textbf{Teiler und Primzahlen} \\
Definition: \\
(i) Es seien $x \in \mathbb{Z}, k \in \mathbb{N}$. k heißt \textit{Teiler} von $x$, geschrieben $k|x$, falls
es ein $q \in \mathbb{Z}$ gibt, so dass $x = k \cdot q$. \\
(ii) Seien $a,b \in \mathbb{N}$. Dann ist der \textit{größte gemeinsame Teiler von $a,b$} definiert durch: \\
$\text{ggT}(a,b) = \text{max}\{k \in \mathbb{N} : k | a \land k | b \}$ \\
(iii) Eine natürliche Zahl $p > 1$ heißt \textit{Primzahl} oder \textit{prim}, wenn sie genau zwei
Teiler besitzt: die 1 und sich selbst.
\bigskip

\textbf{Hauptsatz der Zahlentheorie} \\
Jede natürliche Zahl $n > 1$ lässt sich, bis auf die Reihenfolge der Faktoren, eindeutig als Produkt von 
Primzahlen darstellen. \\

Folgerung: Jeder gemeinsame Teiler von $a$ und $b$ ist auch Teiler von ggT$(a,b)$.
\bigskip

\textbf{Satz:} Seien $a,b,k \in \mathbb{N}, x,y \in \mathbb{Z}$. Dann gilt: \\
Aus $k |a$ und $k|b$ folgt $k | (a x + b y)$.
\bigskip

\textbf{Satz:} Seien $a, b,c \in \mathbb{N}$, $x,y \in \mathbb{Z}$. Dann gilt: \\
Besitzt die Gleichung  $a x + b y = c$ eine Lösung $(x/y)$, so folgt: ggT$(a,b) | c$.
\bigskip

\textbf{Satz:} (Teilen mit Rest) Seien $a, b \in \mathbb{N}$. \\
Dann gibt es eindeutig bestimmte Zahlen $q, r \in \mathbb{N}_0$ mit: $a = q \cdot b + r$ und $0 \le r \le b-1$
\bigskip

\textbf{Satz:} Seien $a, b \in \mathbb{N}, q,r \in \mathbb{N}_0$ und $a = q \cdot b + r$ . Dann gilt:\\
$\text{ggT}(a,b) = \text{ggT}(b,r)$
\bigskip

\textbf{Satz:} (Erweiterter Euklidscher Algorithmus) \\
$\forall a, b \in \mathbb{N} \, \exists x, y \in \mathbb{Z}: ax + by = \text{ggT}(a,b)$
\bigskip

\textbf{Satz:} (Lösungen diophantischer Gleichungen) \\
Gegeben sei die diophantische Gleichung $ax+by=c$ mit $\text{ggT}(a,b)|c$. Dann gilt: \\
 (i)  Es gibt mindestens eine Lösung $(x_0,y_0)$ \\
(ii) Ist $(x_0,y_0)$ eine Lösung, dann sind auch alle
Zahlenpaare $(x_0+k b/y_0-k a)$ mit $k \in \mathbb{Z}$ Lösungen. \\
(iii) Gilt $\text{ggT}(a,b) = 1$, dann sind durch (ii) alle Lösungen gegeben. 
\bigskip

\textbf{Kongruenz} \\
Definition: Seien $a, b \in \mathbb{Z} ,m \in \mathbb{N}$. \\
Dann heißt \textit{a kongruent zu b modulo m}, falls $a-b$ durch $m$ teilbar ist. \\ Wir schreiben dann: $\Mod{a}{b}{m}$.
\bigskip

\textbf{Satz:} Folgende Aussagen sind äquivalent: \\
(1) $\Mod{a}{b}{m}$. \\
(2) $\exists k \in \mathbb{Z}: a = b + k \cdot m$ \\
(3) Beim Teilen mit Rest a durch m, b durch m bleibt derselbe Rest. 
\bigskip

\textbf{Satz:} Die Relation \textit{kongruent modulo m} ist eine Äquivalenzrelation auf $\mathbb{Z}$. \\
(1) $\Mod{a}{a}{m}$ \quad (Reflexivität) \\
(2) $\Mod{a}{b}{m} \Rightarrow \Mod{b}{a}{m}$ \quad (Symmetrie) \\
(3) $\Mod{a}{b}{m}$ und $\Mod{b}{c}{m} \Rightarrow \Mod{a}{c}{m}$ \quad (Transitivität) 
\bigskip

\textbf{Satz:} (Rechenregeln für Kongruenzen) \\
Wenn  $\Mod{a}{b}{m}$ und  $\Mod{c}{d}{m}$, dann gilt: \\
(1)  $\Mod{-a}{-b}{m}$  \\
(2)  $\Mod{a+c}{b+d}{m}$ \\
(3)  $\Mod{a\cdot c}{b \cdot d}{m}$ \\
(4)  $\Mod{a^2}{b^2}{m},  \Mod{a^3}{b^3}{m},$ etc. \\

\newpage

\textbf{Restklassen} \\
Definition: Die \textit{Restklasse $\overline{a}$ von $a$ modulo m} ist definiert durch: \\
$\overline{a}: = \{b \in \mathbb{Z}: \Mod{b}{a}{m} \}$ \\
Andere Schreibweise für die Restklasse $\overline{a}$: $[a]$
\bigskip

\textbf{Rechnen im Restklassenring} \\
Definition: Seien $a, b \in \mathbb{Z}$. \\
$\overline{a} + \overline{b} := \overline{a+b}$ \\
$\overline{a} \cdot \overline{b} := \overline{a \cdot b}$ 
\bigskip

\textbf{Satz:} (Satz vom Dividieren) \\
Ist $p$ eine Primzahl und sind $a \in \mathbb{Z}, b \in \{1,...,p-1\}$, so besitzt die Gleichung \\
$\overline{b} \cdot \overline{x} = \overline{a}$ in $\mathbb{Z}_p$ genau eine Lösung $\overline{x}$,
d.h. $\dfrac{\overline{a}}{\overline{b}}$ ist definiert.
\bigskip

\textbf{Merkregel:} \\
Wenn wir $\dfrac{\overline{1}}{\overline{a}}$ in $\mathbb{Z}_m$ suchen, dann lösen
wir die diophantische Gleichung $ax+my = 1$. \\
Es gilt dann: $\dfrac{\overline{1}}{\overline{a}} = \overline{x}$
\bigskip

\textbf{Der kleine Satz von Fermat:} \\
Sei $p$ Primzahl, $a \in \mathbb{N}$ kein Vielfaches von $p$. Dann gilt: \\
$\Mod{a^{p-1}}{1}{p}$
\bigskip

\textbf{Primitivwurzel} \\
Definition: Ein Element $\overline{g} \in \mathbb{Z}_m$ heißt \textit{Primitivwurzel}, falls durch
$\overline{g}^k$ alle Elemente von $\mathbb{Z}_m$ außer $\overline{0}$ dargestellt werden können.
\bigskip

\textbf{Diffie-Hellman Schlüsselaustausch} \\
Alice und Bob vereinbaren Primzahl $p$ und $g \in \{1,...,p-1\}$ (am besten eine Primitivwurzel). \\
Alice wählt geheim eine Zahl $a$ aus, Bob geheim eine Zahl $b$ mit $a,b \in \{1,...,p-1\}$. \\
Alice berechnet $A = g^a \mod p$, Bob berechnet $B = g^b \mod p$. \\
Dann tauschen Sie $A$ und $B$ aus. Öffentlich bekannt sind also $p, g, A, B$. \\
Beide können nun den gemeinsamen Schlüssel $K$ berechnen: \\
Alice rechnet $K = B^a \mod p$, Bob rechnet $K = A^b \mod p$
\bigskip

\textbf{RSA-Verfahren} \\
Alice wählt zwei Primzahlen $p, q$ und berechnet $m = p \cdot q$ und $\tilde{m} = (p-1)  (q-1).$ \\
Alice wählt Verschlüsselungsexponent $e$ mit $1 < e < \tilde{m}$ und $\ggT(e,\tilde{m}) = 1$. \\
Alice berechnet den Entschlüsselungsexponent $d$ mit: $ \Mod{e \cdot d}{1}{\tilde{m}}$ \\
$(m,e)$ ist der öffentliche Schlüssel, $(m,d)$ der private Schlüssel von Alice. \\
Bob verschlüsselt die Nachricht $n, (0 < n < m)$: $N = n^e \mod m$ \\
Alice entschlüsselt die Nachricht: $n = N^d \mod m$








 

 


 


 



 









\end{document}
