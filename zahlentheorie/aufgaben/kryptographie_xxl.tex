\documentclass[landscape,twocolumn,a4paper]{article}
\usepackage[utf8x]{inputenc}
\usepackage[ngerman]{babel}
\usepackage{listings}
\usepackage{babel}

\usepackage[T1]{fontenc}
\usepackage{booktabs} % schöne Tabellen
\usepackage{graphicx}
\usepackage{csquotes} % Anführungszeichen
\usepackage{paralist} % kompakte Aufzählungen
\usepackage{amsmath,textcomp,tikz} %diverses
\usepackage{eso-pic} % Bilder im Hintergrund
\usepackage{mdframed} % Boxen
\usepackage{multirow}
\usepackage{amssymb}

\usepackage{mathtools}
\usepackage[top=20mm,left=10mm,right=10mm,bottom=10mm]{geometry}
\usepackage{fancyhdr}
\pagestyle{fancy}
\fancyhead[L]{Aufgaben zu Zahlentheorie/Kryptographie}
\fancyhead[R]{\thepage}
\fancyfoot{}

\lstset{language=Python, tabsize=4, basicstyle=\footnotesize, showstringspaces=false, mathescape=true}
\lstset{literate=%
  {Ö}{{\"O}}1
  {Ä}{{\"A}}1
  {Ü}{{\"U}}1
  {ß}{{\ss}}1
  {ü}{{\"u}}1
  {ä}{{\"a}}1
  {ö}{{\"o}}1
}
\begin{document}
\newcommand{\ggT}{\operatorname{ggT}}
\newcommand{\Mod}[3]{#1\equiv#2\mod#3}
\newcommand\x{1}
\newcounter{y}
\setcounter {y} {1}

\parindent 0mm



\textbf{Diophantische Gleichungen} 
\bigskip

%\textbf{A\arabic {y}:}   \\
% Versuche, ganzzahlige Lösungen $(x/y)$ zu finden. Falls du
% vermutest, dass es keine Lösung gibt, begründe deine Vermutung. \\
% a. $x+3y=10$ \quad b. $3x+7y=1$ \quad c. $18x+12y=3$ \\
% d. $5x+5y=1$ \quad e. $5x+15y=50$ \quad f. $18x+12y=66$
% \bigskip \stepcounter{y}
% 
% \textbf{A\arabic {y}:}   \\
%  Untersuche, ob $\text{ggT}(a,b)$ Teiler von
%  $c$ ist, und rate gegebenenfalls mindestens eine Lösung $(x/y)$: \\
% a. $x+3y=10$ \quad b. $3x+7y=1$ \quad c. $18x+12y=3$ \\
% d. $5x+5y=1$ \quad e. $5x+15y=50$ \quad f. $18x+12y=66$
% \bigskip \stepcounter{y}
%
% \textbf{A\arabic {y}:}   \\
%  Teile $a$ durch $b$ mit Rest und schreibe die Lösung als Gleichung
%  $a=kb+r$ auf. \\
% a. $a=143,\ b=12$ \quad b. $a=14130,\ b=58$ \\
% c. $a=1\:111\:111,\ b=2\:222$ \quad d. $a=123\:321,\ b=2\,010$
% \bigskip \stepcounter{y}
%
%\textbf{A\arabic {y}:}   \\
%Bestimme durch Probieren mehrere ganzzahlige Lösungen, möglichst alle.: \\
%a.  $3x+2y=1$ \quad b. $3x+9y=3$ 
% \bigskip \stepcounter{y}
 
\textbf{A\arabic {y}:}   \\
Berechne mit dem Euklidischen Algorithmus: \\
a.  $\ggT(150,54)$ \quad b. $\ggT(300,468)$ \quad 
 c.$\ggT(2717,2431)$ \quad d. $\ggT(4263,4641)$
\bigskip \stepcounter{y}
 
\textbf{A\arabic {y}:}   \\
Berechne den ggT der Zahlen $a$ und $b$ und stelle ihn in der Form $ax + by$ dar. \\
a.   $ a = 531, b = 93$  \quad b. $ a = 753, b = 64$
\bigskip \stepcounter{y}

\textbf{A\arabic {y}:}   \\
Bestimme eine Lösung $(x/y)$ der angegebenen Gleichung: \\
a.  $96x+66y=6$ \quad b.$96x+66y=18$ \\
 c. Für beliebiges fest vorgegebenes $n\in\mathbb{N}$:\quad $96x+66y=n\cdot 6$ \\
 d.  $119x+143y=4$ \quad e. $91x+35y=12$.
\bigskip \stepcounter{y}
 
 
 \textbf{A\arabic {y}:}   \\
Vereinfache die Gleichung und finde möglichst viele Lösungen: \\
a.   $42x+126y=84$ \quad b. $81x+54y=27$ \quad 
 c. $77x+121y=44$ 
 \bigskip \stepcounter{y}
 
 \textbf{A\arabic {y}:}   \\
  Ein zerstreuter Bankkassierer verwechselte 1-Euromünzen und
  1-Centmünzen, als er den Scheck von Herrn Krause auszahlte, indem er
  ihm 1-Euromünzen anstelle von 1-Centmünzen und 1-Centmünzen anstelle
  von 1-Euromünzen gab. Nachdem Herr Krause zuhause großzügig 5 Cent
  in die Spardose seines Sohnes getan hatte, entdeckte er, dass er
  jetzt noch genau doppelt so viel Geld hatte, wie auf dem Scheck
  stand. Auf welche Summe war der Scheck ausgestellt?
 \bigskip \stepcounter{y}
 
 \textbf{A\arabic {y}:}   \\
  Gib alle natürlichen Zahlen an, die  bei Division durch
  $19$ den Rest $3$ und gleichzeitig bei Division durch $29$ den Rest
  $18$ lassen.
 \bigskip \stepcounter{y}
 
 \textbf{Kongruenzen} \bigskip
 
\textbf{A\arabic {y}:}   \\
Bestimme den Rest beim Teilen durch 9. \\
a. $1000$ \quad b.   $2005$ \quad  c. $5103$  \\
d.   Bestimme eine 7-stellige Zahl, die beim Teilen durch 9 den Rest 3
\bigskip \stepcounter{y}

\textbf{A\arabic {y}:}   \\
Bestimme möglichst alle ganzzahligen Lösungen $x$ der folgenden
  Gleichungen: \\
a. $\Mod{5+x}{2}{7}$\quad b.   $\Mod{5\cdot x}{2}{7}$ \\ 
c. $\Mod{5\cdot x}{2}{10}$  \quad d.  $\Mod{-34}{x}{5}$
\bigskip \stepcounter{y}

\textbf{A\arabic {y}:}   \\
Beweise die folgenden Aussagen:\\
a. Wenn $\Mod{a}{b}{m}$ und $\Mod{c}{d}{m}$, dann
    $\Mod{a+c}{b+d}{m}$. \\
b. Wenn $\Mod{a}{b}{m}$, dann $\Mod{-a}{-b}{m}$. \\
c. Wenn $\Mod{a}{b}{m}$ und $\Mod{b}{c}{m}$, dann
    $\Mod{a}{c}{m}$.
\bigskip \stepcounter{y}

\textbf{A\arabic {y}:}   \\
  Welche der folgenden Gleichungen sind garantiert falsch? (Ohne
  Taschenrechner!) \\
a. ${12345}\cdot{54321}\ \overset{?}{=}\ {670592745}$ \quad b.   $6613598\cdot55500710\ \overset{?}{=}\ 367359384654580$ \\
c. $6613598\cdot55500710\ \overset{?}{=}\ 367059384654580$ \\
d. $6613598\cdot55500710\cdot432 \overset{?}{=}\ 158569654170778570$ \\
e.  $123456709+6789402+878787487+1232123\overset{?}{=}1010365721$ \\
f. $123456709+6789402+878787487+1232123\overset{?}{=}1010265721$
\bigskip \stepcounter{y}

\textbf{A\arabic {y}:}   \\
Bestimme möglichst alle ganzzahligen Lösungen $x$ der folgenden
  Gleichungen: \\
a. $\Mod{5+x}{2}{7}$\quad b.   $\Mod{5\cdot x}{2}{7}$ \\ 
c. $\Mod{5\cdot x}{2}{10}$  \quad d.  $\Mod{-34}{x}{5}$
\bigskip \stepcounter{y}
\end{document}
