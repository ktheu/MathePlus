\documentclass[landscape,twocolumn,a4paper]{article}
\usepackage[utf8x]{inputenc}
\usepackage[ngerman]{babel}
\usepackage{listings}
\usepackage{babel}

\usepackage[T1]{fontenc}
\usepackage{booktabs} % schöne Tabellen
\usepackage{graphicx}
\usepackage{csquotes} % Anführungszeichen
\usepackage{paralist} % kompakte Aufzählungen
\usepackage{amsmath,textcomp,tikz} %diverses
\usepackage{eso-pic} % Bilder im Hintergrund
\usepackage{mdframed} % Boxen
\usepackage{multirow}
\usepackage{amssymb}

\usepackage{mathtools}
\usepackage[top=20mm,left=10mm,right=10mm,bottom=10mm]{geometry}
\usepackage{fancyhdr}
\pagestyle{fancy}
\fancyhead[L]{Reihen}
\fancyhead[R]{\thepage}
\fancyfoot{}

\lstset{language=Python, tabsize=4, basicstyle=\footnotesize, showstringspaces=false, mathescape=true}
\lstset{literate=%
  {Ö}{{\"O}}1
  {Ä}{{\"A}}1
  {Ü}{{\"U}}1
  {ß}{{\ss}}1
  {ü}{{\"u}}1
  {ä}{{\"a}}1
  {ö}{{\"o}}1
}
\begin{document}
\newcounter{y}
\setcounter {y} {1}
\parindent 0mm


\textbf{Reihen}
\bigskip 

\textbf{A\arabic {y}:}   
Bestimme den Grenzwert der Reihen \\
a. $\sum\limits_{k=0}^\infty (\dfrac{1}{10})^k$ \quad
b. $\sum\limits_{k=1}^\infty \dfrac{2^k+(-3)^k}{5^k}$ \quad
c. $\sum\limits_{k=2}^\infty \dfrac{2^{k+1}}{7 \cdot 5^k}$  \quad
d. $-\frac{1}{2} + \frac{1}{6} - \frac{1}{18} + \frac{1}{54} - \frac{1}{162} ...$
\bigskip  \stepcounter{y}

\textbf{A\arabic {y}:}  
Wandle in einen Bruch um: \\
a. $0.\overline{48}$ \quad b. $3.1\overline{48}$ \quad c. $0.\overline{1234}$
\bigskip \stepcounter{y}
 

\textbf{A\arabic {y}:}   
Bestimme den Grenzwert der Reihen (Hinweis: Teleskopreihen) \\
a. $\sum\limits_{k=1}^\infty \dfrac{1-\frac{1}{\pi}}{\pi^k}$ \quad
%b. $\sum\limits_{k=1}^\infty \dfrac{1}{4k^2-1}$ \quad
b. $\sum\limits_{k=1}^\infty (\frac{1}{k}-\frac{1}{k+3})$ \quad
\bigskip \stepcounter{y}

\textbf{A\arabic {y}:}   
Zeige durch Vergleich mit der harmonischen Reihe: \quad
$\sum\limits_{k=1}^\infty \dfrac{1}{\sqrt{k}}= \infty$  
\bigskip \stepcounter{y}

\textbf{A\arabic {y}:}   
Sind die Reihen konvergent? \\
a. $\sum\limits_{n=0}^\infty (\frac{\cos n}{3})^n$   \quad
b. $\sum\limits_{n=0}^\infty \frac{n^2}{3^n}$   \quad
c. $\sum\limits_{n=0}^\infty \frac{x^n}{n!}$   \quad
\bigskip \stepcounter{y}


\textbf{A\arabic {y}:}   
Bestimme den Konvergenzradius der Potenzreihen \\
a. $\sum\limits_{n=0}^\infty \frac{k+2}{2^k}x^k$   \quad
b. $\sum\limits_{n=0}^\infty \frac{(2+x)^{2k}}{(2+\frac{1}{k})^k}$   \quad
c. $\sum\limits_{n=0}^\infty \frac{3^{k+2}}{2^k}x^k$   \quad
\bigskip \stepcounter{y}

\textbf{A\arabic {y}:}   
Bestimme Taylorreihe und Konvergenzradius um $x_0 =0$ \\
a. $f(x) = e^{-x}$   \quad b. $f(x) = e^{x^2}$ \quad c. $f(x) = \ln(1-\frac{x}{2})$ \quad d.$f(x) = \frac{1}{1+x}$
\bigskip \stepcounter{y}

\textbf{A\arabic {y}:}   
Bestimme das Taylorpolynom $p_5$ \\
a. $f(x) = \sqrt{1-x}$  \\
b. $f(x) = \arcsin(x)$ \quad \textit{Hinweis:} $f'(x) = \frac{1}{\sqrt{1-x^2}}$ \\
c. $f(x) = \sinh(x) = \dfrac{e^x-e^{-x}}{2}$ 

\bigskip \stepcounter{y}
\textbf{A\arabic {y}:} 
Bestimme die Taylorreihe für folgende Funktionen um $x_0$. \\
a. $f(x) = x e^x, x_0 = 1$  \quad
b. $f(x) =\cosh(x) = \dfrac{e^x+e^{-x}}{2}, x_0 = 0$ 
 
\bigskip 


 



\end{document}
