\documentclass[11pt]{beamer}
\usepackage[ngerman]{babel}
\usepackage[utf8]{inputenc}
\usepackage{amsmath}
\usepackage{amssymb}
\usepackage{listings} 
\usepackage{stmaryrd}
\lstset{language=Python, tabsize=4, showstringspaces=false,basicstyle=\footnotesize,mathescape=true} 
\lstset{literate=%
  {Ö}{{\"O}}1
  {Ä}{{\"A}}1
  {Ü}{{\"U}}1
  {ß}{{\ss}}1
  {ü}{{\"u}}1
  {ä}{{\"a}}1
  {ö}{{\"o}}1
}
\usepackage{mathtools}
\usepackage{ulem}
\usepackage{tikz}

\usetheme{Boadilla}
\mode<presentation>{
\useoutertheme[subsection=false]{miniframes}
\useinnertheme{rectangles}
%\usecolortheme{crane}
}
\parskip 10pt



\begin{document}
\title{Vertiefungskurs Mathematik}   
\author{Reihen, Potenzreihen, Taylorreihen} 
\date{}
\frame{\titlepage} 


%---
\begin{frame}[fragile]
Reihen entstehen, wenn man die Elemente einer Folge aufsummiert.

Sei $(a_n)$ eine Folge. Wir bilden die Partialsummen $s_n = \sum\limits_{n=0}^\infty a_n$.
Wen $



\end{frame}


 

 


\end{document}


