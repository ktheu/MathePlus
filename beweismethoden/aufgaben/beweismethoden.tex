\documentclass[landscape,twocolumn,a4paper]{article}
\usepackage[utf8x]{inputenc}
\usepackage[ngerman]{babel}
\usepackage{listings}
\usepackage{babel}

\usepackage[T1]{fontenc}
\usepackage{booktabs} % schöne Tabellen
\usepackage{graphicx}
\usepackage{csquotes} % Anführungszeichen
\usepackage{paralist} % kompakte Aufzählungen
\usepackage{amsmath,textcomp,tikz} %diverses
\usepackage{eso-pic} % Bilder im Hintergrund
\usepackage{mdframed} % Boxen
\usepackage{multirow}
\usepackage{amssymb}

\usepackage{mathtools}
\usepackage[top=20mm,left=10mm,right=10mm,bottom=10mm]{geometry}
\usepackage{fancyhdr}
\pagestyle{fancy}
\fancyhead[L]{Aufgaben zu Beweismethoden}
\fancyhead[R]{\thepage}
\fancyfoot{}

\lstset{language=Python, tabsize=4, basicstyle=\footnotesize, showstringspaces=false, mathescape=true}
\lstset{literate=%
  {Ö}{{\"O}}1
  {Ä}{{\"A}}1
  {Ü}{{\"U}}1
  {ß}{{\ss}}1
  {ü}{{\"u}}1
  {ä}{{\"a}}1
  {ö}{{\"o}}1
}
\begin{document}

\newcommand\x{1}
\newcounter{y}
\setcounter {y} {1}

\parindent 0mm

\textbf{Direkter Beweis}
\bigskip

\textbf{A\arabic {y}:} Zeige: Teilt t die natürlichen Zahlen a und b, dann teilt t auch deren Summe.
\bigskip \stepcounter{y}
 
\textbf{A\arabic {y}:} Zeige: Es gibt unendlich viele Primzahlen.
\bigskip \stepcounter{y}

\textbf{A\arabic {y}:} Zeige: Teilt t die natürlichen Zahlen a > b, so auch deren Differenz a − b.
\bigskip \stepcounter{y}

\textbf{A\arabic {y}:} Beweise die folgenden Teilbarkeitsregeln. Alle Zahlen seien aus $\mathbb{N}$. \\
a. Ist a ein Teiler von b, und teilt b wiederum c, so ist auch a ein Teiler von c. \\
b. Wenn gilt: a teilt c und b teilt d, dann teilt a $\cdot$ b das Produkt c $\cdot$ d. \\
c. Teilt t die Zahlen a und b, dann teilt t auch m $\cdot$ a + n $\cdot$ b.
\bigskip \stepcounter{y}

\textbf{A\arabic {y}:} Stelle Vermutungen über Summen bzw. Produkte gerader und ungerader Zahlen auf (ob
diese wieder gerade oder ungerade sind) und beweise sie anschließend.
\bigskip \stepcounter{y}

\textbf{A\arabic {y}:} Zeige: Ist die Quersumme einer Zahl durch 3 (9) teilbar, dann
ist auch die Zahl selbst durch 3 (9) teilbar. (Es genügt, die Beweisidee an einem
Beispiel zu entwickeln.)
\bigskip \stepcounter{y}

\textbf{A\arabic {y}:} Zeige: Die Gleichung a $\cdot$ x = b mit a, b $\in \mathbb{N}$ ist genau dann in $\mathbb{N}$
lösbar, wenn b ein Vielfaches von a ist.
\bigskip \stepcounter{y}

\textbf{A\arabic {y}:} Zeige: Für alle $a,b > 0$ gilt die Ungleichung $\frac{2}{\frac{1}{a}+\frac{1}{b}} \le \sqrt{ab}$
\bigskip \stepcounter{y}

\textbf{A\arabic {y}:} Es sei $n \ge 2$ eine natürliche Zahl. Zeige: $n$ ist genau dann ungerade,
wenn sich n als Differenz von Quadraten zweier aufeinanderfolgender natürlicher Zahlen darstellen lässt.
\bigskip \stepcounter{y}

\textbf{Indirekter Beweis}
\bigskip

\textbf{A\arabic {y}:} Beweise durch Kontraposition: Wenn $n^2$ gerade ist (für ein $n \in \mathbb{N}$), dann ist auch $n$ gerade.
\bigskip \stepcounter{y}

\textbf{A\arabic {y}:} Beweise durch Kontraposition, dass zwei aufeinanderfolgende natürliche Zahlen teilerfremd sind.
\bigskip \stepcounter{y}

\textbf{A\arabic {y}:} Beweise durch Kontraposition, Ist eine Zahl gerade, so ist ihre
letzte Ziffer (im Zehnersystem) gerade.
\bigskip \stepcounter{y}

\textbf{A\arabic {y}:} Finde Beispiele für einen wahren Satz $A \Rightarrow B$, für den weder
$\lnot A \Rightarrow \lnot B$ noch sein Kehrsatz $B \Rightarrow A$ wahr sind (warum ist eine der beiden
Forderungen überflüssig?). Überzeuge dich zudem davon, dass die Kontraposition
des Satzes wahr ist.
\bigskip \stepcounter{y}



\textbf{Widerspruchsbeweis}
\bigskip 

\textbf{A\arabic {y}:} Beweise durch Widerspruch, dass zwei aufeinanderfolgende natürliche Zahlen teilerfremd sind.
\bigskip \stepcounter{y}

\textbf{A\arabic {y}:} Führe einen Widerspruchsbeweis, um die Ungleichung $2 \cdot \sqrt{ab} \le a + b$ zu 
beweisen.
\bigskip \stepcounter{y}

\textbf{A\arabic {y}:} Beweise, dass $\sqrt{3}$ irrational ist (allgemeiner: $\sqrt{p}$ für $p$ prim).
\bigskip \stepcounter{y}

\textbf{A\arabic {y}:} Beweise durch Widerspruch, dass es unendlich viele Primzahlen gibt.
\bigskip \stepcounter{y}

\textbf{A\arabic {y}:} Zeige: Es gibt keine Primzahl, die als Differenz von Quadraten nicht aufeinanderfolgender natürlicher Zahlen dargestellt werden kann.
\bigskip \stepcounter{y}

\textbf{Vollständige Induktion}
\bigskip 

\textbf{A\arabic {y}:} Für jedes $n \in \mathbb{N}$ gilt die
arithmetische Summenformel: 

$ 1 + 2 + 3 + ... + n = \frac{1}{2}n(n+1)$.
\bigskip \stepcounter{y}

\textbf{A\arabic {y}:}  Für jedes reelle $x$ mit $0 \ne x > -1$
und alle $n \in \mathbb{N}, n \ge 2$ gilt die Bernoulli-Ungleichung

$ (1 + x)^n > 1 +nx$.
\bigskip \stepcounter{y}

\textbf{A\arabic {y}:}  Für jedes $n \in \mathbb{N}$ ist 8 ein Teiler von $9^n - 1$.
\bigskip \stepcounter{y}

\textbf{A\arabic {y}:} Jede natürliche Zahl n > 1 ist ein Produkt von Primzahlen.
\bigskip \stepcounter{y}

\textbf{A\arabic {y}:} Für alle $n \in \mathbb{N}$ gelten die folgenden Summenformeln.

a. $1 + 4 + 7 + ... + (3n - 2) = \frac{1}{2}n(3n - 1)$ \\
b. $1 + 3 + 5 + ... + (2n - 1) = n^2$ \\
c. $1^2 + 2^2 + ... + n^2 = \frac{1}{6}n(n+1)(2n+1)$ \\
d. $1^3 + 2^3 + ... + n^3 = \frac{1}{4}n^2(n+1)^2$ \\

Folgere aus d. und der arithmetischen Summenformel: \\
$(1 + 2  + ... + n)^2 = 1^3 + 2^3 + ... + n^3$
\bigskip \stepcounter{y}

\textbf{A\arabic {y}:} Für $n \in \mathbb{N}$ und $1 \ne q \in \mathbb{R}$ gilt die geometrische Summenformel

$1 + q + q^2 + ... + q^n = \frac{1- q^{n+1}}{1-q}$
\bigskip \stepcounter{y}

\textbf{A\arabic {y}:} Beweise die folgenden Teilbarkeitsregeln.

a. $9$ ist Teiler von $10^n-1$ für alle $n \in \mathbb{N}$. (Wie sieht man das ohne Induktion?) \\
b. $6$ ist ein Teiler von $n^3 - n$ für alle $n \in \mathbb{N}$ mit $n \ge 2$. Zerlege $n^3-n$ in Linearfaktoren, 
um dies auch ohne vollständige Induktion einzusehen.
\bigskip \stepcounter{y}

\textbf{A\arabic {y}:}  \\
a. Leite die Funktion $f(x) = \frac{1}{x}$ ein paar Mal ab. Stelle eine Vermutung
für die n-te Ableitung $f^{(n)}(x)$ auf und beweise sie. \\
b. Zeige, dass für die n-te Ableitung von $f(x) = \frac{1}{\sqrt{x}}, x > 0$, gilt: \\
$f^{(n)}(x) = \frac{1 \cdot 3 \cdot 5 \cdot ... \cdot (2n-1)}{(-2)^n \cdot \sqrt{x}^{2n+1}}$

\bigskip \stepcounter{y}

\textbf{A\arabic {y}:}  \\
In einem Klassenzimmer befinden sich $n$ Schüler, die sich alle mit Handschlag begrüßen. Zeige,
dass dabei $\frac{1}{2}(n-1)\cdot n$ Handschläge stattfinden.
\bigskip \stepcounter{y}

\textbf{A\arabic {y}:}  \\
Zeige durch vollständige Induktion: Für alle $n \in \mathbb{N}, n \ge 2$ gilt:

$\sum\limits_{k=1}^{n-1} \frac{2k+1}{k^2(k+1)^2} = 1 - \frac{1}{n^2}.$
\bigskip \stepcounter{y}
\end{document}
