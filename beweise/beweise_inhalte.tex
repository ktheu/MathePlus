\documentclass[a4paper]{article}
\usepackage[utf8x]{inputenc}
\usepackage[ngerman]{babel}
\usepackage{listings}
\usepackage{babel}

\usepackage[T1]{fontenc}
\usepackage{booktabs} % schöne Tabellen
\usepackage{graphicx}
\usepackage{csquotes} % Anführungszeichen
\usepackage{paralist} % kompakte Aufzählungen
\usepackage{amsmath,textcomp,tikz} %diverses
\usepackage{eso-pic} % Bilder im Hintergrund
\usepackage{mdframed} % Boxen
\usepackage{multirow}
\usepackage{amssymb}

\usepackage{mathtools}
\usepackage[top=20mm,left=10mm,right=10mm,bottom=10mm]{geometry}
\usepackage{fancyhdr}
\pagestyle{fancy}
\fancyhead[L]{Beweise}
\fancyhead[R]{\thepage}
\fancyfoot{}

\lstset{language=Python, tabsize=4, basicstyle=\footnotesize, showstringspaces=false, mathescape=true}
\lstset{literate=%
  {Ö}{{\"O}}1
  {Ä}{{\"A}}1
  {Ü}{{\"U}}1
  {ß}{{\ss}}1
  {ü}{{\"u}}1
  {ä}{{\"a}}1
  {ö}{{\"o}}1
}
\begin{document}

\parindent 0mm

Mathematische Sätze sind meist als $A \Rightarrow B$ formuliert. Lies: 'Aus A folgt B' oder 'A impliziert B' oder 'Wenn A wahr ist, dann ist auch B wahr'. \\
 
Man nennt dann $A$ die \textit{Voraussetzung} und $B$ die \textit{Behauptung}. 

\bigskip

\textbf{Direkter Beweis}
\bigskip

Beim \textit{direkten} Beweis geht man davon aus, dass $A$ wahr ist und folgert durch eine Kette gültiger Argumente, dass dann auch $B$ wahr ist. 
\bigskip

\textbf{Indirekter Beweis}
\bigskip

Beim \textit{indirekten Beweis} zeigt man die Kontraposition:
 $\lnot B \Rightarrow \lnot A$. Man geht davon aus, dass $B$ falsch ist und folgert durch eine Kette
 gültiger Argumente, dass dann auch $A$ falsch ist.
\bigskip
 
\textbf{Widerspruchsbeweis}
\bigskip

 Man geht davon aus, dass die zu beweisende Aussage falsch ist
und führt dies zu einem Widerspruch, z.B. zu einem absurden Resultat wie $1 < 0$, oder etwa dass gleichzeitig $A$ und $\lnot A$ wahr ist. Somit erweist sich die Annahme, die zu beweisende Aussage sei falsch, als nicht haltbar, woraus die Wahrheit der Aussage folgt.
\bigskip

\textbf{Vollständige Induktion}
\bigskip

 Die vollständige Induktion ist ein Beweisverfahren, das in folgender Situation angewendet wird: 
 Zu jeder natürlichen Zahl $n$ ist eine Aussage
$A(n)$ gegeben, deren Gültigkeit man beweisen will.

Die Aussagen $A(n)$ sind für alle $n \in \mathbb{N}$ wahr, wenn man Folgendes zeigen kann:

- (IA) Induktionsanfang: $A(1)$ ist wahr. \\
- (IS) Induktionsschritt: Wenn $A(n)$ wahr ist, dann ist auch $A(n+1)$ wahr.

Die Annahme, dass $A(n)$ wahr ist, heißt Induktionsvoraussetzung (IV).


\end{document}









\end{document}
