\documentclass[landscape,twocolumn,a4paper]{article}
\usepackage[utf8x]{inputenc}
\usepackage[ngerman]{babel}
\usepackage{listings}
\usepackage{babel}

\usepackage[T1]{fontenc}
\usepackage{booktabs} % schöne Tabellen
\usepackage{graphicx}
\usepackage{csquotes} % Anführungszeichen
\usepackage{paralist} % kompakte Aufzählungen
\usepackage{amsmath,textcomp,tikz} %diverses
\usepackage{eso-pic} % Bilder im Hintergrund
\usepackage{mdframed} % Boxen
\usepackage{multirow}
\usepackage{amssymb}

\usepackage{mathtools}
\usepackage[top=20mm,left=10mm,right=10mm,bottom=10mm]{geometry}
\usepackage{fancyhdr}
\pagestyle{fancy}
\fancyhead[L]{Beweise}
\fancyhead[R]{\thepage}
\fancyfoot{}

\lstset{language=Python, tabsize=4, basicstyle=\footnotesize, showstringspaces=false, mathescape=true}
\lstset{literate=%
  {Ö}{{\"O}}1
  {Ä}{{\"A}}1
  {Ü}{{\"U}}1
  {ß}{{\ss}}1
  {ü}{{\"u}}1
  {ä}{{\"a}}1
  {ö}{{\"o}}1
}
\begin{document}

\parindent 0mm


\textbf{Aufgaben der Zertifikatsklausuren}
\bigskip 

\textbf{A2021} \\
a) Beweisen Sie mit vollständiger Induktion, dass \\$\sum\limits_{j=0}^n (2j+1) = (n+1)^2$ für alle $n \in \mathbb{N}$ gilt. \\
b) Es seien $a, b$ reelle Zahlen. Beweisen Sie: Für alle $n \in \mathbb{N}$ gilt \\
$(a-b) \cdot \sum\limits_{j=0}^n a^j b^{n-j} = a^{n+1} - b^{n+1}$.


\bigskip

\textbf{A2020} \\
Es sei $f: (0,\infty) \rightarrow \mathbb{R}$ eine Funktion mit der Eigenschaft \\
(*) \quad $f(x\cdot y) = f(x) + f(y)$ für alle $x,y > 0$. \\
a) Beweisen Sie mit vollständiger Induktion, dass $f(2^n) = n \cdot f(2)$ für $n \in \mathbb{N}$ gilt. \\
b) Beweisen Sie, dass $f(1) = 0$ gilt. \\
c) Beweisen Sie, dass $f(\frac{1}{x}) = -f(x)$ für alle $x > 0$ gilt. \\

\textit{Hinweise:} Setzen Sie geeignete Werte für $x,y$ in (*) ein. Im Aufgabenteil c) darf das Ergebnis aus b)
verwendet werden, auch wenn Teil b) nicht gelöst wurde.
\bigskip

\textbf{A2019} \\
Für $n \in \mathbb{N}$ und $x \in \mathbb{R}$ sei $p_n(x) = 1 - x^{2^n}$. \\
a) Beweisen Sie durch vollständige Induktion, dass \\
$p_n(x) = (1-x)(1+x)(1+x^2)(1+x^4)...(1+x^{2^{n-1}})$ für alle $n \in \mathbb{N}$ mit $n \ge 2$ gilt. \\
b) Folgern Sie aus a), dass $p_n$ außer $x = \pm 1$ keine weiteren Nullstellen besitzt ($n \in \mathbb{N})$.
\bigskip

\textbf{A2018} \\
a) Gegeben sind vier positive reelle Zahlen $a_1, a_2, b_1, b_2 > 0$ mit der Eigenschaft
$\dfrac{a_1}{b_1} \le \dfrac{a_2}{b_2}$. Beweisen Sie, dass dann 
$\dfrac{a_1}{b_1} \le \dfrac{a_1+a_2}{b_1+b_2} \le \dfrac{a_2}{b_2}$ gilt. \\
b) Gegeben sind $2n$ positive Zahlen  $a_1, a_2,...,a_n b_1, b_2,...,b_n > 0$ mit der Eigenschaft \\
$\dfrac{a_1}{b_1} \le \dfrac{a_2}{b_2} ... \le \dfrac{a_{n-1}}{b_{n-1}} \le \dfrac{a_n}{b_n}$. \\
 Beweisen Sie, dass  
$\dfrac{a_1}{b_1} \le \dfrac{a_1+a_2+...+a_n}{b_1+b_2+...+b_n} \le \dfrac{a_n}{b_n}$ 
\bigskip

\newpage

\textbf{A2017} \\
Es sei $n \in \mathbb{N}$ mit $n \ge 3$. Ein einfaches n-Eck hat n verschiedene Eckpunkte $E_1,...,E_n$, die
durch Kanten verbunden sind. Außerdem schneiden sich die Kanten nicht, und für die Innenwinkel 
$\alpha_1,...,\alpha_n$ an den Ecken gilt: $\alpha_j \neq 0^\circ, 180^\circ, 360^\circ$ für $j=1,...,n$. \\
Beweisen Sie durch vollständige Induktion, dass für die Summe $S_n$ der Innenwinkel in einem einfachen
n-Eck mit $n \ge 3$ gilt: $S_n = (n-2) \cdot 180^\circ$. Hierbei darf ohne Beweis verwendet werden,
dass die Winkelsumme im Dreieck $180^\circ$ beträgt. \\
\textit{Hinweis:} Unterscheiden Sie im Induktionsschritt die Fälle $\alpha_{n+1} > 180^\circ$ und
$\alpha_{n+1} < 180^\circ$. Verwenden Sie die in der jeweiligen Skizze eingezeichnete Hilfslinie.

\includegraphics[scale=0.8]{bild2.png} \\
\bigskip

\newpage
\textbf{A2016} \\
In dieser Aufgabe beschäftigen wir uns mit den Sechseckszahlen $H_n (n=0,1,2,...)$. Wir betrachten
dazu Anordnungen von Kreisen mit gleichem Radius, die schrittweise folgendermaßen erzeugt werden:
Im Schritt 0 beginnen wir mit einem einzelnen Kreis, der im Schritt 1 wie unten skizziert durch
Anlagerung von sechs weiteren Kreisen zu einer sechseckartigen Figur ergänzt wird. Nachfolgend
wird im Schritt n + 1 die Figur aus dem Schritt n durch eine weitere äußere Lage von Kreisen zu
einer noch größeren sechseckartigen Figur ergänzt, wobei sich die Länge der äußeren Seiten um jeweils
eine Kugel erhöht. Die Sechseckzahl $H_n$ entspricht der Gesamtzahl der Kugeln in der so erzeugten
Figur im Schritt n. \\

\includegraphics[scale=0.8]{bild1.png} \\

a) Drücken Sie $H_{n+1}$ durch $H_n$ aus $(n=0,1,2,...)$ \\
b) Zeigen Sie mit vollständiger Induktion, dass die n-te Sechseckzahl die Gleichung \\
 $H_n =3n^2 + 3n +1$ für 
$n \in \mathbb{N}$ erfüllt. \\
c) Zeigen Sie, dass die Summe der Sechseckzahlen gerade die Kubikzahlen
 $\sum\limits_{k=0}^{n-1} H_k = n^3$ liefert $n \in \mathbb{N}$. \textit{Hinweis:} Sie dürfen die Formel aus
 Teil b) verwenden, auch wenn Sie diese nicht bewiesen haben.
\bigskip
\newcounter{y}
\setcounter {y} {1}

\newpage
\textbf{Sonstige Aufgaben}
\bigskip

\textbf{A\arabic {y}:}
Identifiziere Voraussetzung und Behauptung. Überlege, ob die Aussage wahr oder falsch ist und beweise oder widerlege sie.

a) Jede natürliche Zahl $n \ge 2$ hat eine gerade Anzahl von Teilern. \\
b) Das Produkt zweier ungerader natürlicher Zahlen ist ungerade. \\
c) Für natürliche Zahlen $n$ gilt: $n^3$ ist ungerade, wenn $n$ ungerade ist. \\
d) $n(n-1)+41$ ist für jede natürliche Zahl $n$ eine Primzahl. \\
e) Für jede natürlicher Zahl $n$ gilt: Ist $n^2 + 6n + 4$ ungerade, dann ist $n$ ungerade.
\bigskip \stepcounter{y}

\textbf{A\arabic {y}:}
Eine Zahl $a \in \mathbb{R}$ heißt rational, wenn sie sich als Bruch $a = \frac{p}{q}$ mit ganzzahligen $p, q$ und 
$q \neq 0$ darstellen lässt. Gegeben seien eine feste rationale Zahl $a \neq 0$ und eine beliebige reelle Zahl $b$. Folgender Satz soll untersucht werden: \\
Ist $b$ nicht rational, so ist auch $a \cdot b$ nicht rational.

a. Gib die Voraussetzung und Behauptung des Satzes an. \\
b. Bilde die Kontraposition. \\
c. Beweise den Satz.
\bigskip \stepcounter{y}

\textbf{A\arabic {y}:} Beweise durch Widerspurch \\
a) $\sqrt{3}$ ist irrational. \\
b) Ist $x$ rational, so ist $\sqrt{2} + x$ irrational.
\bigskip \stepcounter{y}

\textbf{A\arabic {y}:}
Beweise mit vollständiger Induktion: Für alle natürlichen Zahlen gilt: \\
a) $\sum\limits_{k=1}^n k^2 = \dfrac{1}{6}n(n+1)(2n+1)$ \\
b) $\sum\limits_{k=1}^n k^3 =   \dfrac{1}{4}n^2(n+1)^2$ \\
c)  $\sum\limits_{k=1}^n 2k =  n(n+1)$ \\
d) 5 ist Teiler von $6^n-1$ \\
e) 6 ist Teiler von $n^3-n$ \\
f) $2^n > n$ \\
g) $n^2 > 2n + 1$ falls $n \ge 3$
\bigskip \stepcounter{y}
 
 
% vollständige Induktion:
%a) $\sum\limits_{k=1}^n (2k-1)^2 = \dfrac{n(2n+1)(2n-1)}{3}$ \\
%b) $5^n +3$ ist durch 4 teilbar. \\
%c) $3^n > 12n$ für $n >=4$

%\newpage
%\textbf{Zusatzaufgaben}
%\bigskip
%
%\textbf{A\arabic {y}:}
%Identifiziere Voraussetzung und Behauptung. Überlege, ob die Aussage wahr oder falsch ist und beweise oder widerlege sie.
%
%a) $\sum\limits_{k=1}^{1000}(n+k)$ ist für keine natürliche Zahl $n$ eine Primzahl. \\
%b) Ist $n^2$ durch 4 teilbar, dann ist auch $n$ durch 4 teilbar. \\
%c) Wenn $p$ und $p+2$ Primzahlen sind, spricht man von einem Primzahlzwilling. Ist $p  \ge 5$ und sind
%$p$ und $p + 2$ Primzahlen, dann ist $p + 4$ keine Primzahl. (D.h. es gibt keine Primzahldrillinge
%$p, p + 2, p + 4$ mit $p \ge 5$.) \\
%d) Es gibt genau eine natürliche Zahl $n$, für die $n-9$ eine Primzahl und $n^2-1$ durch 10 teilbar
%ist. \\
%e) Es gibt genau eine Primzahl $p$, für die $4p + 1$ eine Quadratzahl ist. 
%\bigskip \stepcounter{y}
%
%
%\textbf{A\arabic {y}:}  
%Die Zahl $3$ ist eine besondere Primzahl: Sie ist die einzige Primzahl, deren nachfolgende natürliche
%Zahl eine Quadratzahl ist. Zu Beweisen ist also: Ist $p \ne 3$ eine Primzahl, dann ist die nachfolgende
%natürliche Zahl keine Quadratzahl. \\
%a) Identifiziere Voraussetzung und Behauptung. \\
%b) Wie lautet die Kontraposition? Verwende die Regel von de Morgan! \\
%c) Beweise die Kontraposition
%\bigskip \stepcounter{y}
\end{document}
