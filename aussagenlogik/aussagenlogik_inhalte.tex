\documentclass[a4paper]{article}
\usepackage[utf8x]{inputenc}
\usepackage[ngerman]{babel}
\usepackage{listings}
\usepackage{babel}

\usepackage[T1]{fontenc}
\usepackage{booktabs} % schöne Tabellen
\usepackage{graphicx}
\usepackage{csquotes} % Anführungszeichen
\usepackage{paralist} % kompakte Aufzählungen
\usepackage{amsmath,textcomp,tikz} %diverses
\usepackage{eso-pic} % Bilder im Hintergrund
\usepackage{mdframed} % Boxen
\usepackage{multirow}
\usepackage{amssymb}

\usepackage{mathtools}
\usepackage[top=20mm,left=10mm,right=10mm,bottom=10mm]{geometry}
\usepackage{fancyhdr}
\pagestyle{fancy}
\fancyhead[L]{Aussagenlogik, Prädikatenlogik}
\fancyhead[R]{\thepage}
\fancyfoot{}

\lstset{language=Python, tabsize=4, basicstyle=\footnotesize, showstringspaces=false, mathescape=true}
\lstset{literate=%
  {Ö}{{\"O}}1
  {Ä}{{\"A}}1
  {Ü}{{\"U}}1
  {ß}{{\ss}}1
  {ü}{{\"u}}1
  {ä}{{\"a}}1
  {ö}{{\"o}}1
}
\begin{document}

\parindent 0mm

\textbf{Aussagenlogik}
\bigskip

Eine \textbf{Aussage} ist ein Satz oder eine Formel, der man genau einen Wahrheitswert zuordnen kann (w: wahr, f: falsch). Befehle oder Fragen sind keine Aussagen. \\
Statt w und f nutzen wir auch 1 und 0. \\

Mit \textbf{Junktoren}  können wir zusammengesetzte Aussagen bilden. \\

- Negation - $\lnot$ - nicht \\
- Konjunktion - $\land$ - und \\
- Disjunktion - $\lor$ - oder (nicht ausschließend) \\
- Kontravalenz - $\oplus$ - xor - entweder...oder - ausschließendes oder - $\dot \lor$, $\veebar$ \\
- Implikation -  $\Rightarrow$ - wenn...dann  \\
- Äquivalenz - $\Leftrightarrow$ - genau dann, wenn  \\
 

\bigskip

\textbf{Wahrheitstafeln} 
\bigskip 
\begin{displaymath}
\begin{array}{|c c|c|c|c|c|c|c|}

p & q & p \land q & p \lor q & p \oplus q & p \Rightarrow q & p \Leftrightarrow q & \lnot p\\ 
\hline  
0 & 0 & 0 & 0 & 0 & 1 & 1 & 1\\
0 & 1 & 0 & 1 & 1 & 1 & 0 & 1\\
1 & 0 & 0 & 1 & 1 & 0 & 0 & 0\\
1 & 1 & 1 & 1 & 0 & 1 & 1 & 0\\
\end{array}
\end{displaymath}

\textbf{Gesetze der Aussagenlogik} 
\bigskip 

\begin{tabular}{l l}
$  p \Leftrightarrow \lnot (\lnot p) $   &   doppelte Negation \\
\\
$p \land q \Leftrightarrow p \land q$ &  Kommutativgesetze\\
$p \lor q  \Leftrightarrow p \lor q$ \\
\\
$(p \land q) \land r \Leftrightarrow p \land (q \land r)$ & Assoziativgesetze \\
$(p \lor q) \lor r \Leftrightarrow p \lor (q \lor r)$ \\
\\
$(p \land q) \lor r\Leftrightarrow (p \lor r) \land (q \lor r)$  & Distributivgesetze \\
$(p \lor q) \land r \Leftrightarrow (p \land r) \lor (q \land r)$ \\
\\
$\lnot(p \land q) \Leftrightarrow \lnot p \lor \lnot q$ & DeMorgansche Regeln \\
$\lnot(p \lor q) \Leftrightarrow \lnot p \land \lnot q$  \\
\\
$p \Rightarrow q \Leftrightarrow \lnot q \Rightarrow \lnot p$ & Kontrapositionsregel \\
\\
$(p \Rightarrow q) \Leftrightarrow (\lnot p \lor q)$ & Sonstige \\
$p \land p  \Leftrightarrow p $ \\
$p \lor p  \Leftrightarrow p $  \\
$p \land \lnot p  \Leftrightarrow 0 $ \\
$p \lor \lnot p  \Leftrightarrow 1 $ \\
\\
$\lnot$ bindet stärker als $\lor$ und $\land$ und diese binden stärker als $\Rightarrow$, $\Leftrightarrow$. 
\end{tabular} \\
\bigskip

\textbf{Prädikatenlogik}
\bigskip

\begin{tabular}{c l}

$\forall x \in X: p(x)$ & Für alle $x$ aus $X$ ist die Aussage $p(x)$ wahr. \\
$\exists x \in X: p(x)$ & Es gibt mindestens ein $x$ aus $X$ für das die Aussage $p(x)$ wahr ist.\\
\end{tabular} \\

\bigskip
\textbf{Prädikatenlogische Verneinungsregeln} \\

$\lnot (\forall x \in X: p(x)) \Leftrightarrow \exists x \in X : \lnot p(x)$   \\
$\lnot (\exists x \in X : p(x)) \Leftrightarrow \forall x \in X: \lnot p(x)$  
\bigskip

Quantoren können auch hintereinander stehen: \\
$\lnot (\forall x \in X \, \exists y \in Y: p(x,y)) \Leftrightarrow \exists x \in X \, \forall y \in Y: \lnot p(x,y)$
\end{document}









\end{document}
