\documentclass[landscape,twocolumn,a4paper]{article}
\usepackage[utf8x]{inputenc}
\usepackage[ngerman]{babel}
\usepackage{listings}
\usepackage{babel}

\usepackage[T1]{fontenc}
\usepackage{booktabs} % schöne Tabellen
\usepackage{graphicx}
\usepackage{csquotes} % Anführungszeichen
\usepackage{paralist} % kompakte Aufzählungen
\usepackage{amsmath,textcomp,tikz} %diverses
\usepackage{eso-pic} % Bilder im Hintergrund
\usepackage{mdframed} % Boxen
\usepackage{multirow}
\usepackage{amssymb}

\usepackage{mathtools}
\usepackage[top=20mm,left=10mm,right=10mm,bottom=10mm]{geometry}
\usepackage{fancyhdr}
\pagestyle{fancy}
\fancyhead[L]{Aufgaben zur Aussagenlogik}
\fancyhead[R]{\thepage}
\fancyfoot{}

\lstset{language=Python, tabsize=4, basicstyle=\footnotesize, showstringspaces=false, mathescape=true}
\lstset{literate=%
  {Ö}{{\"O}}1
  {Ä}{{\"A}}1
  {Ü}{{\"U}}1
  {ß}{{\ss}}1
  {ü}{{\"u}}1
  {ä}{{\"a}}1
  {ö}{{\"o}}1
}
\begin{document}

\parindent 0mm

\textbf{Aufgaben aus den Zertifikatsklausuren}
\bigskip

\textbf{A2020} \\

a) Beweisen Sie mit Hilfe einer Wahrheitstabelle, dass die Aussage $ \lnot(A \land B) \Leftrightarrow (\lnot A \lor \lnot B)$ für beliebige Wahrheitswerte von A und B wahr ist. \\


b) Tim, Chris, Niko und Alex tragen T-Shirts in verschiedenen Farben: Schwarz, weiß, grün und
blau. Vanessa hat die drei gesehen und erzählt ihren Freundinnen: \\
(1) Tims Shirt ist nicht schwarz und auch nicht weiß. \\
(2) Das Shirt von Alex ist blau. \\
(3) Tim trägt nicht blau und Niko nicht weiß. \\
Die Freundinnen raten sofort los: Alex trägt blau, Tim grün, Niko schwarz und Chris weiß.
Aber Vanessa ergänzt, dass alle drei Aussagen (1), (2) und (3) falsch sind. \\

b1) Verneinen Sie die Aussagen (1), (2) und (3). \\
b2) Beweisen Sie, dass eindeutig bestimmt ist, welcher der vier Jungs welche Farbe trägt,
wenn die drei Aussagen falsch sind. Geben Sie an, wer welche Farbe trägt.
\bigskip 

\textbf{A2019} \\

a) Beweisen Sie mit Hilfe einer Wahrheitstabelle, 
dass die Aussage $(A \Rightarrow B) \Leftrightarrow (\lnot A \lor B) $
für beliebige Wahrheitswerte von A und B wahr ist. \\

b) Ein Geheimdienst beobachtet vier Spione A, B, C, D und möchte ihre Namen herausbekommen.
Nachdem ein Treffen der vier Spione beobachtet wurde, steht fest, dass ihre Namen
Alexander, Francois, James und Pjotr sind, und dass keine zwei den selben Namen besitzen.
Außerdem konnte ermittelt werden, dass die folgenden Aussagen wahr sind. \\

b1) A heißt James oder Alexander, \\
b2) Wenn A James heißt, dann heißt C Francois, \\
b3) Wenn B nicht Alexander heißt, dann heißt C Pjotr, \\
b4) C heißt nicht Francois, \\
b5) B heißt Pjotr oder B heißt nicht Francois. \\
Zeigen Sie, dass die Namen der Spione A, B, C, D durch
die Angaben eindeutig bestimmt
sind, und geben Sie an, wie jede der Personen heißt.
\bigskip 

\newpage

\textbf{A2018} \\

a) Beweisen Sie mit Hilfe einer Wahrheitstabelle, dass die folgende Aussage für beliebige Wahrheitswerte
von A und B wahr ist:  $(A \Leftrightarrow B) \Leftrightarrow ((A \land B) \lor (\lnot A \land \lnot B))$. \\

b) Vor einem Fußballturnier fachsimpeln Zuschauer über den möglichen Ausgang. Über die drei Favoriten A, B und C 
werden folgende vier Vermutungen geäußert: \\

a. B gewinnt oder C gewinnt. \\
b. Wenn B Zweiter wird, dann gewinnt A. \\
c. Wenn B Dritter wird, dann gewinnt C  nicht. \\
d. A wird Zweiter oder B wird Zweiter.

Am Ende des Turniers belegen die drei Favoriten tatsächlich die ersten drei Plätze. Es stellt sich heraus, dass 
alle vier Vermutungen richtig waren. Welche Plätze erzielten A, B, und C?
\bigskip 

\textbf{A2017} \\

a) Beweisen Sie mit Hilfe einer Wahrheitstabelle, dass die Aussage
$\lnot (A \Rightarrow B) \Leftrightarrow (A \land \lnot B)$ 
für beliebige Wahrheitswerte von A, B wahr ist. \\

b) Bei einem Ausflug unterhält sich eine Gruppe von Schülern, es ist von Aki, Bauzi, Chips und
Dani die Rede. Die Lehrerin möchte wissen, wovon die Schüler reden. Ein Schüler antwortet:
\glqq Von einem Jungen, einem Mädchen, einem Hund und einer Katze.\grqq\, Außerdem bekommt die
Lehrerin folgende Hinweise: \\
(1) Die Katze heißt Chips. \\
(2) Aki ist nicht der Junge und Bauzi ist nicht das Mädchen. \\
(3) Wenn Dani das Mädchen ist, dann ist Aki der Hund. \\
Die Lehrerin meint, da gäbe es mehrere Möglichkeiten. Daraufhin lachen die Schüler los, und
einer sagt: \glqq Die Hinweise sind alle falsch. \grqq\, Nach kurzem Überlegen weiß die Lehrerin Bescheid. \\

b1) Zeigen Sie, dass die Bedingungen (1), (2), (3) mindestens zwei verschiedene Zuordnungen
der Namen zulassen. \\
b2) Verneinen Sie die Aussagen (1), (2) und (3). 
\textit{Hinweis:} Die Äquivalenz aus Teil a) darf verwendet werden. \\
b3) Wie heißen der Junge, das Mädchen, die Katze und der Hund? Weisen Sie nach, dass
die Lösung eindeutig ist.
\bigskip 

\newpage

\textbf{A2016} \\

a) Beweisen Sie mit Hilfe einer Wahrheitstabelle, dass die Aussage
$(A \land ( \lnot B \Rightarrow \lnot A)) \Rightarrow B$
für beliebige Wahrheitswerte von A, B wahr ist. \\

b) Der Kommissar hat drei Tatverdächtige: Paula, Quentin und Ralf. Er weiß: \\
a. Wenn sich Quentin oder Ralf als Täter herausstellen, ist Paula unschuldig. \\
b. Ist aber Paula oder Ralf unschuldig, dann muss Quentin ein Täter sein. \\
c. Ist Ralf schuldig, so ist Paula Mittäterin.

Wer ist schuldig? Wer ist unschuldig?
\bigskip


\textbf{Sonstige Aufgaben}
\bigskip

\newcounter{y}
\setcounter {y} {1}


\textbf{A\arabic {y}:}
Beweisen Sie, dass folgende Aussagen für beliebige Wahrheitswerte von A und B wahr sind: \\

a. $[((\lnot A) \Rightarrow B) \land (\lnot B)] \Rightarrow A$ \\
b. $(A \Rightarrow B) \Leftrightarrow (\lnot A \lor B)$
\bigskip \stepcounter{y}

\textbf{A\arabic {y}:} 
Formuliere den Satz aussagenlogisch.

a. Wenn Peter nicht kommt und wenn Otto nicht kommt, dann kommt Clara nicht. \\
(p: Peter kommt, q: Otto kommt, r: Clara kommt) 

b. Wenn der Hahn kräht auf dem Mist, ändert sich das Wetter oder es bleibt wie es ist. \\
(p: Der Hahn kräht, q: Das Wetter ändert sich)

c. Die Mutter ist wütend genau dann, wenn Klaus nicht sein Zimmer aufräumt oder Wilma die Schule schwänzt. \\
(m: Mutter ist wütend, k: Klaus räumt Zimmer auf, w: Wilma schwänzt die Schule
\bigskip \stepcounter{y}

\textbf{A\arabic {y}:} 
Gegeben seien natürliche Zahlen a, b und c.
Formuliere zu folgendem Satz die Umkehrung, die Kontraposition und die Kontraposition der Umkehrung. \\
Wenn a und  b durch c teilbar sind, dann ist auch ihre Summe a+b durch c teilbar.
\bigskip \stepcounter{y}

\newpage
\textbf{A\arabic {y}:}
Gegeben sind die folgenden Aussagen:

a. Jeder Drache ist nicht grün \\
b. Für jeden Drachen gilt: Wenn er fliegen kann, dann ist er glücklich. \\
c. Alle grünen Drachen können fliegen und sind glücklich. \\
d. Wenn ein Drache grün ist, dann sind alle seine Kinder grün. \\
e. Jeder Drache ist glücklich, wenn alle seine Kinder fliegen können. \\

Formuliere die Aussagen prädikatenlogisch. Benutze dafür $X:=$ die Menge aller Drachen, $K(x):=$ Menge aller 
Kinder des Drachen $x$, und die Aussagen $fl(x), gl(x), gr(x)$, deren Wahrheitswerte folgendermaßen definiert sind:

- Die Aussage $fl(x)$ ist genau dann wahr, wenn der Drache $x$ fliegen kann, \\
- Die Aussage $gl(x)$ ist genau dann wahr, wenn der Drache $x$ glücklich ist, \\
- Die Aussage $gr(x)$ ist genau dann wahr, wenn der Drache $x$ grün ist.
\bigskip \stepcounter{y}

\end{document}
