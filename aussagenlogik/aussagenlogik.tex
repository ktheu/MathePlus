\documentclass[landscape,twocolumn,a4paper]{article}
\usepackage[utf8x]{inputenc}
\usepackage[ngerman]{babel}
\usepackage{listings}
\usepackage{babel}

\usepackage[T1]{fontenc}
\usepackage{booktabs} % schöne Tabellen
\usepackage{graphicx}
\usepackage{csquotes} % Anführungszeichen
\usepackage{paralist} % kompakte Aufzählungen
\usepackage{amsmath,textcomp,tikz} %diverses
\usepackage{eso-pic} % Bilder im Hintergrund
\usepackage{mdframed} % Boxen
\usepackage{multirow}
\usepackage{amssymb}

\usepackage{mathtools}
\usepackage[top=20mm,left=10mm,right=10mm,bottom=10mm]{geometry}
\usepackage{fancyhdr}
\pagestyle{fancy}
\fancyhead[L]{Aufgaben zur Logik}
\fancyhead[R]{\thepage}
\fancyfoot{}

\lstset{language=Python, tabsize=4, basicstyle=\footnotesize, showstringspaces=false, mathescape=true}
\lstset{literate=%
  {Ö}{{\"O}}1
  {Ä}{{\"A}}1
  {Ü}{{\"U}}1
  {ß}{{\ss}}1
  {ü}{{\"u}}1
  {ä}{{\"a}}1
  {ö}{{\"o}}1
}
\begin{document}

\newcommand\x{1}
\newcounter{y}
\setcounter {y} {1}

\parindent 0mm

\textbf{Aussagenlogik}

\bigskip

\textbf{A\arabic {y}:} Handelt es sich um eine Aussage?  

a. $1 + 1 = 3.$  \\
b. Gehen wir Mammuts jagen? \\
c. Urrkh fragt Ankk, ob sie Mammuts jagen gehen. \\
d. Halt den Mund, Rotzbub, frecher!  \\
e. Der Lehrer fordert den Schüler höflich auf, die Privatgespräche einzustellen.
\bigskip \stepcounter{y}
 

\textbf{A\arabic {y}:} Handelt es sich um eine Aussage?
 
a. Dieser Satz ist falsch \\
b. Ich lüge gerade.  \bigskip \stepcounter{y}

\textbf{A\arabic {y}:} 
Erstelle die Wahrheitstafeln für folgende Ausdrücke:

a. $(p \rightarrow q) \lor (q \rightarrow (p \land q))$ \\
b. $((p \rightarrow q) \lor q) \rightarrow (p \land q)$ \\
c. $\lnot a \lor b \leftrightarrow \lnot b$
\bigskip \stepcounter{y}

\textbf{A\arabic {y}:} 
Vereinfache die folgenden Ausdrücke.

a. $\lnot p \land q \rightarrow p \lor q$ \\
b. $\lnot p \land (p \lor q)$ \\
c. $p \lor q \rightarrow \lnot p \land \lnot q$ \\
d. $\lnot (\lnot q \rightarrow \lnot p) \rightarrow \lnot p$ \\
%e. $(\lnot p \land q) \lor \lnot(p \lor q)$
\bigskip \stepcounter{y}

\textbf{A\arabic {y}:} 
Entscheide, ob die folgenden Äquivalenzen wahr oder falsch sind.

a. $\lnot(\lnot p \lor q) \Leftrightarrow p \land \lnot q$ \\
b. $p \land \lnot (p \rightarrow q) \Leftrightarrow p \land \lnot q$ \\
c. $p \lor \lnot q \Leftrightarrow \lnot ( \lnot p \land \lnot q)$ \\
d. $((p \lor q) \land \lnot p \rightarrow q) \Leftrightarrow (p \lor \lnot p)$ \\
e. $p \lor (q \land r) \Leftrightarrow (p \lor q) \land r$ 
\bigskip \stepcounter{y}

\textbf{A\arabic {y}:} 
Bilde mit Hilfe der DeMorgan-Regeln die Verneinung der beiden folgenden Aussagen zur Abendgestaltung eines Schülers. 

a. Gustl lernt Mathe und geht pumpen. \\
b. Gustl lernt Mathe oder geht pumpen.
\bigskip \stepcounter{y}

\textbf{A\arabic {y}:} 
Beweise die Regel: $\lnot (a \rightarrow b) \Leftrightarrow a \land \lnot b$. Verneine damit: Wenn Gustl Mathe lernt, dann geht er nicht pumpen.
\bigskip \stepcounter{y}



\textbf{A\arabic {y}:} 
In dieser Aufgabe soll gezeigt werden, dass sich alle bisher eingeführten Junktoren allein durch die zwei Junktoren
$\lnot$ und $\lor$ ausdrücken lassen.

a. Zeige, dass für die Subjunktion gilt: $a \rightarrow b \Leftrightarrow \lnot a \lor b$. \\
b. Gewinne eine  \{$\lnot$,$\lor$\}-Darstellung des $\land$-Junktors durch Verneinung der ersten DeMorgan-Regel. \\
c. Finde für die Bijunktion $\leftrightarrow$ und Kontravalenz $\oplus$ entsprechende Ausdrücke.
\bigskip \stepcounter{y}

\textbf{A\arabic {y}:} 
Finde für die Negation der Kontravalenz eine  \{$\lnot$,$\lor$\}-Darstellung
\bigskip \stepcounter{y}

\textbf{A\arabic {y}:} 
Auf der Postkarte eines Mathematikers steht: "Jedes Mal, wenn es geregnet hat, kamen Außerirdische und 
klauten unser Zelt."\,  Was ist gemeint?
\bigskip \stepcounter{y}

\textbf{A\arabic {y}:} 
Formuliere den Satz aussagenlogisch.

a. Wenn Peter nicht kommt und wenn Otto nicht kommt, dann kommt Clara nicht. \\
(p: Peter kommt, q: Otto kommt, r: Clara kommt) 

b. Wenn der Hahn kräht auf dem Mist, ändert sich das Wetter oder es bleibt wie es ist. \\
(p: Der Hahn kräht, q: Das Wetter ändert sich)

c. Die Mutter ist wütend genau dann, wenn Klaus nicht sein Zimmer aufräumt oder Wilma die Schule schwänzt. \\
(m: Mutter ist wütend, k: Klaus räumt Zimmer auf, w: Wilma schwänzt die Schule
\bigskip \stepcounter{y}

\textbf{A\arabic {y}:} 
Wir betrachten für ein Würfelexperiment mit zwei Würfeln die folgenden Aussagen: \\
a: „Der erste Würfel zeigt eine ungerade Augenzahl“. \\
b: „Der zweite Würfel zeigt eine gerade Augenzahl“.  \\

Formuliere die folgenden Aussagen als aus a und b zusammengesetzte Aussagen. \\
c: „Die Summe der Augenzahlen ist eine gerade Zahl“. \\
d: „Die Differenz der Augenzahlen ist eine ungerade Zahl“. \\
e: „Das Produkt der Augenzahlen ist eine ungerade Zahl“.
\bigskip \stepcounter{y}


 
\textbf{A\arabic {y}:} 
Durch die folgende Wahrheitstafel wird eine neue logische Verknüpfung $*$ definiert:

\begin{tabular}{|c|c||c|}
\hline   p & q & p $*$ q  \\
\hline w  & w & f  \\
\hline w  & f & f  \\
\hline f & w & f \\
\hline f & f & w \\
\hline
\end{tabular}

a. Stelle die Wahrheitstafel von $p * p$ auf. \\
b. Welche einfache logische Verknüpfung hat dieselbe Wahrheitstabelle wie $p * p$?
\bigskip \stepcounter{y}

\textbf{A\arabic {y}:} 
Formuliere zu folgendem Satz die Umkehrung, die Kontraposition und die Kontraposition der Umkehrung. \\
Wenn zwei natürliche Zahlen a, b durch dieselbe Zahl c ($c\in\mathbb{N}$) teilbar sind, dann ist auch ihre Summe a+b durch c teilbar.
\bigskip \stepcounter{y}

\textbf{A\arabic {y}:} 
Der Kommissar hat drei Tatverdächtige: Paula, Quentin und Ralf. Er weiß: \\
a. Wenn sich Quentin oder Ralf als Täter herausstellen, ist Paula unschuldig. \\
b. Ist aber Paula oder Ralf unschuldig, dann muss Quentin ein Täter sein. \\
c. Ist Ralf schuldig, so ist Paula Mittäterin.

Wer ist schuldig? Wer ist unschuldig?
\bigskip \stepcounter{y}

\textbf{A\arabic {y}:} 
Einer der vier Herren Krause, Lehmann, Meier und Schulze ist von Beruf Arzt, ein anderer Ingenieur, ein dritter Lehrer und der vierte Notar. Welchen Beruf übt jeder dieser vier aus, wenn die drei folgenden Aussagen falsch sind?

a. Herr Meier ist nicht Lehrer und nicht Ingenieur. \\
b. Herr Meier ist nicht Notar und Herr Schulze nicht Ingenieur. \\
c. Herr Lehmann ist Notar.
\bigskip \stepcounter{y}

\textbf{A\arabic {y}:} 
Auf der Insel Wafa leben zwei Stämme: die Was, die immer die Wahrheit sagen, und die Fas, die immer lügen. Ein Reisender besucht die Insel und trifft auf drei Einwohner A, B, C, die ihm folgendes erzählen: \\

- A sagt: B und C sagen genau dann die Wahrheit, wenn C die Wahrheit sagt. \\
- B sagt: Wenn A und C die Wahrheit sagen, dann ist es nicht der Fall, dass A die Wahrheit sagt, wenn B und C die Wahrheit sagen. \\
- C sagt: B lügt genau dann, wenn A oder B die Wahrheit sagen.

Welchen Stämmen gehören A, B und C an?
\bigskip \stepcounter{y} 

\textbf{A\arabic {y}:} 
Fred möchte mit möglichst vielen seiner Freunde Anne, Bernd, Christine,
Dirk und Eva seinen Geburtstag feiern. Er weiß Folgendes:
Wenn Bernd und Anne beide zur Party kommen, dann wird Eva auf keinen
Fall kommen. Und Dirk wird auf keinen Fall kommen, wenn Bernd und Eva
beide zur Feier kommen. Aber Eva kommt allenfalls dann, wenn Christine
und Dirk kommen. Andererseits kommt Christine nur dann, wenn auch
Anne kommt. Anne wiederum wird nur dann kommen, wenn auch Bernd
oder Christine dabei sind.
Wie viele Freunde (und welche) werden im besten Fall zur Party
kommen?
\bigskip \stepcounter{y}



\textbf{Prädikatenlogik}
\bigskip

\textbf{A\arabic {y}:} 
Gegeben sind die einstelligen Prädikate mit den Symbolen M: ' ...  ist ein Mann'  und S: '  ... ist ein 
Schwein'. Formuliere die folgenden prädikatenlogischen Aussagen in Worten. (Welche Aussagen sind wahr?)

a. $\forall x : (M(x) \rightarrow S(x))$ \\
b. $\forall x : (M(x) \land S(x))$ \\
c. $\exists x : (M(x) \land S(x))$ 

\bigskip \stepcounter{y}

\textbf{A\arabic {y}:} 
Wandle die folgenden Aussagen in prädikatenlogische Form um.

a. Alles ist Eins. \\
b. Alle Wege führen nach Rom. \\
c. Einige Schüler sind gut in Mathe. \\
d. Keine Gurke ist eine Tomate (Oder: Alle Gurken sind keine Tomaten.)
\bigskip \stepcounter{y}

 

\textbf{A\arabic {y}:} 
Gegeben sind die folgenden Aussagen:

a. Jeder Drache ist nicht grün \\
b. Für jeden Drachen gilt: Wenn er fliegen kann, dann ist er glücklich. \\
c. Alle grünen Drachen können fliegen und sind glücklich. \\
d. Wenn ein Drache grün ist, dann sind alle seine Kinder grün. \\
e. Jeder Drache ist glücklich, wenn alle seine Kinder fliegen können. \\

Formuliere die Aussagen prädikatenlogisch. Benutze dafür $X:=$ die Menge aller Drachen, $K(x):=$ Menge aller 
Kinder des Drachen $x$, und die Aussagen $fl(x), gl(x), gr(x)$, deren Wahrheitswerte folgendermaßen definiert sind:

- Die Aussage $fl(x)$ ist genau dann wahr, wenn der Drache $x$ fliegen kann, \\
- Die Aussage $gl(x)$ ist genau dann wahr, wenn der Drache $x$ glücklich ist, \\
- Die Aussage $gr(x)$ ist genau dann wahr, wenn der Drache $x$ grün ist.
\bigskip \stepcounter{y}


\textbf{A\arabic {y}:} 
$S(x,y)$ stehe für 'x ist schwerer als y'. Drücke in Worten aus:

a. $\forall x \exists y : S(x,y)$ \\
b. $\exists x \forall y : S(x,y)$ \\
c. $\exists x \exists y: S(x,y)$ 

\bigskip \stepcounter{y}

\textbf{A\arabic {y}:} 
Im Englischen gibt es eine Redensart, die lautet:

\textit{You can fool all people some of the time and you can fool some
people all of the time, but nobody can fool all people all of the
time.}

Bringe dies auf prädikatenlogische Form. Dabei soll $F(x,y,t)$  für das Prädikat
'x can fool y at time t'  stehen und 'you' als generalisierender Ausdruck im
Sinne von 'jeder' aufgefasst werden.

\bigskip \stepcounter{y}

\textbf{A\arabic {y}:} 
Die Definition für den Grenzwert einer Folge lautet:

$\forall \epsilon > 0 \enspace \exists n_{\epsilon}  \in \mathbb{N} 
 \enspace \forall n > n_{\epsilon} : \left| a - a_n \right| < \epsilon $

Drücke dies in Worten aus und negiere die Aussage (formal und in Worten).

\bigskip \stepcounter{y}
\end{document}
