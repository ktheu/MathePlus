\documentclass[a4paper]{article}
\usepackage[utf8x]{inputenc}
\usepackage[ngerman]{babel}
\usepackage{listings}
\usepackage{babel}

\usepackage[T1]{fontenc}
\usepackage{booktabs} % schöne Tabellen
\usepackage{graphicx}
\usepackage{csquotes} % Anführungszeichen
\usepackage{paralist} % kompakte Aufzählungen
\usepackage{amsmath,textcomp,tikz} %diverses
\usepackage{eso-pic} % Bilder im Hintergrund
\usepackage{mdframed} % Boxen
\usepackage{multirow}
\usepackage{amssymb}

\usepackage{mathtools}
\usepackage[top=20mm,left=10mm,right=10mm,bottom=10mm]{geometry}
\usepackage{fancyhdr}
\pagestyle{fancy}
\fancyhead[L]{Gleichungen und Ungleichungen}
\fancyhead[R]{\thepage}
\fancyfoot{}

\lstset{language=Python, tabsize=4, basicstyle=\footnotesize, showstringspaces=false, mathescape=true}
\lstset{literate=%
  {Ö}{{\"O}}1
  {Ä}{{\"A}}1
  {Ü}{{\"U}}1
  {ß}{{\ss}}1
  {ü}{{\"u}}1
  {ä}{{\"a}}1
  {ö}{{\"o}}1
}
\begin{document}

\parindent 0mm

\textbf{Lösungsmengen}
\bigskip

Lösungen zu Gleichung und Ungleichungen geben wir in Mengenschreibweise oder mittels Intervallen an.
\bigskip

\textbf{Ganzrationale Gleichungen} 
\bigskip

Eine Funktion $p:\mathbb{R} \rightarrow \mathbb{R}, \,
p(x) = a_0 + a_1 x^1 + a_2 x^2 + ... + a_n x^n$ \\
mit $a_i \in \mathbb{R}, a_n \neq 0$ heißt Polynom n-ter Ordnung.

\bigskip
Ein Polynom $p$ hat  eine Nullstelle in $x_0$ genau dann, wenn es ein Polynom $g$ gibt mit: \\
$p(x) = (x-x_0)g(x)$. Den Term $(x-x_0)$ nennen wir einen \textit{Linearfaktor} von $p$.
\bigskip

Ein Polynom n-ter Ordnung hat maximal n Nullstellen.
\bigskip

Zur Bestimmung der Nullstellen von Polynomen höherer Ordnung kann man oft eine der Nullstellen raten und
durch \textit{Polynomdivision} eine Gleichung niederen Grades erhalten.
\bigskip

Der Graph eines Polynoms kann häufig durch 
die Betrachtung des globalen Verhaltens und der Nullstellen konstruiert werden.
\bigskip

\textbf{Bruchgleichungen} 
\bigskip

An den Nullstellen des Nenners ist ein Bruchterm nicht definiert. Die Nennernullstellen sind also nicht in der Definitionsmenge enthalten.
Errechnete Lösungen, die nicht in der Definitionsmenge liegen, gehören nicht zur Lösungsmenge.
\bigskip

Ein Bruchterm ist Null, wenn der Zähler Null ist.
\bigskip

Beim Lösen einer Bruchgleichung wird meist mit einem gemeinamen Nenner multipliziert.
\bigskip

Nullstellen des Nenners, die keine Nullstellen des Zählers sind, sind im Graphen der Funktion Polstellen.
Das Verhalten des Graphen können wir durch Punktproben in der Nähe der Polstellen bestimmen.
\bigskip

Bei einer Partialbruchzerlegung der Form $\dfrac{cx+d}{(x+a)(x+b)} = \dfrac{A}{x+a} + \dfrac{B}{x+b}$ können wir 
$A$ und $B$ mittels Koeffizientenvergleich oder mittels \textit{Betrachtung der explodierenden Terme} bestimmen.


\bigskip
 

\textbf{Betragsgleichungen}
\bigskip

$\vert a \vert$ lässt sich als Abstand von a zum Nullpunkt auf interpretieren. \\
$\vert a - b \vert$ lässt sich als Abstand zwischen $a$ und $b$ interpretieren.
\bigskip

In Betragsgleichungen lösen wir die Beträge durch Fallunterscheidung auf. Die verschiedenen Fälle erhalten wir, indem
wir betrachten, wann die Terme in den Beträgen ihre Vorzeichen ändern.
\bigskip


\textbf{Wurzelgleichungen}
\bigskip

Das Quadrieren einer Gleichung ist keine Äquivalenzumformung. Deshalb muss am Ende des Lösungsweges eine Probe gemacht werden.
\bigskip
 
\textbf{Ungleichungen}
\bigskip

Auch mit Ungleichungen können wir mit Äquivalenzumformungen durchführen. 
Bei Multiplikation oder Division mit einer negativen Zahl dreht sich das Ungleichheitszeichen um.
\bigskip







\end{document}
