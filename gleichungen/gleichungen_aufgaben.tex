\documentclass[landscape,twocolumn,a4paper]{article}
\usepackage[utf8x]{inputenc}
\usepackage[ngerman]{babel}
\usepackage{listings}
\usepackage{babel}

\usepackage[T1]{fontenc}
\usepackage{booktabs} % schöne Tabellen
\usepackage{graphicx}
\usepackage{csquotes} % Anführungszeichen
\usepackage{paralist} % kompakte Aufzählungen
\usepackage{amsmath,textcomp,tikz} %diverses
\usepackage{eso-pic} % Bilder im Hintergrund
\usepackage{mdframed} % Boxen
\usepackage{multirow}
\usepackage{amssymb}

\usepackage{mathtools}
\usepackage[top=20mm,left=10mm,right=10mm,bottom=10mm]{geometry}
\usepackage{fancyhdr}
\pagestyle{fancy}
\fancyhead[L]{Gleichungen und Ungleichungen}
\fancyhead[R]{\thepage}
\fancyfoot{}

\lstset{language=Python, tabsize=4, basicstyle=\footnotesize, showstringspaces=false, mathescape=true}
\lstset{literate=%
  {Ö}{{\"O}}1
  {Ä}{{\"A}}1
  {Ü}{{\"U}}1
  {ß}{{\ss}}1
  {ü}{{\"u}}1
  {ä}{{\"a}}1
  {ö}{{\"o}}1
}
\begin{document}

\parindent 0mm


\textbf{Aufgaben der Zertifikatsklausuren}
\bigskip 

\textbf{A2021} \\
In dieser Aufgabe werden Polynomfunktionen 3. Grades der Form 

$p(x) = x^3+ax^2+bx+c \quad (*)$ 

betrachtet. Die reellen Konstanten $a, b, c$ heißen Koeffizienten.

a. Gegeben sind $x_1 = 1, x_2 = -1$ und $x_3=3$. Konstruieren Sie die Polynomfunktion 3. Grades
der Form (*), welche die Nullstellen $x_1, x_2,x_3$ besitzt. 

b. Gegeben sind drei (nicht notwendig verschiedene) reelle Zahlen $x_1, x_2,x_3$. Konstruieren Sie die
Polynomfunktion 3. Grades der Form (*), welche die Nullstellen $x_1, x_2,x_3$ besitzt. Geben Sie
jeweils eine Formel an, mit der $a$ bzw. $b$ bzw. $c$ aus den Nullstellen$x_1, x_2,x_3$ berechnet werden
kann.

c. Gegeben ist eine Polynomfunktion $p$ der Form (*) mit ganzzahligen Koeffzienten  $a , b, c$. Außerdem
 ist bekannt, dass $x = 2$ eine Nullstelle von $p$ ist. Beweisen Sie, dass dann der Koeffzient
$c$ durch 2 teilbar ist. \\
\textit{Hinweis:} Eine ganze Zahl $d$ ist durch 2 teilbar, wenn es eine ganze Zahl $k$ gibt, so dass $d = 2k$.

\bigskip

\textbf{A2020} \\
Gegeben sind die Funktionen $f$ und $h$ mit $f(x)=(x+4)^2$ und $h(x)= \frac{1}{5}(x^2-4)$ für $x \in \mathbb{R}$. \\
a) Berechnen Sie die Nullstellen von $f$ und $h$ und die Schnittpunkte der Graphen von $f$ und $h$. \\
b) Skizzieren Sie die Graphen $y = f(x)$ und $y=h(x)$, ihre Schnittpunkte und die Nullstellen von $f$ und $h$ in
einem geeigneten Koordinatensystem. \\
c) Bestimmend Sie die Lösungsmenge der Ungleichung $\frac {1}{5}(x^2-4) \le (x+4)^2$. \\
d) Bestimmen Sie die Lösungsmenge der Ungleichung $\sqrt{\frac {1}{5}(x^2-4)} \le x+4$.
\bigskip

\textbf{A2019} \\
a) Bestimmen Sie die Lösungsmenge der Ungleichung $\dfrac{4x-5}{(x+1)(x-2)} \le 0$. \\
b) Bestimmen Sie reelle Zahlen $A, B$, so dass \\
$\dfrac{4x-5}{(x+1) (x-2)} = \dfrac{A}{x+1} + \dfrac{B}{x-2}$ für alle $x \in \mathbb{R} \setminus \{-1,2\}$
erfüllt ist. \\
c) Skizzieren Sie den Graphen der Funktion $f$ mit \\
$f(x) = \dfrac{4x-5}{(x+1) (x-2)} $ für $x \in \mathbb{R} \setminus \{-1,2\}$ \\
unter Berücksichtigung der Nullstellen, des Monotonieverhaltens und der Asymptoten.
\bigskip

\textbf{A2018} \\
a) Beweisen Sie, dass die Polynomfunktion $p(x) = 6x^2-12x+7$ für alle reellen Werte von $x$ 
positive Werte annimmt. \\
b) Gegeben sind die zwei Gleichungen \\
\quad $\sqrt{6x^2-12x+7} = 3x-2$ \quad (1) \\
\quad $\sqrt{6x^2-12x+7} = 2-3x$ \quad (2) \\
Untersuchen Sie beide Gleichungen auf Lösbarkeit und bestimmen Sie gegebenenfalls alle Lösungen. \\
c) Bestimmen Sie, für welche reellen Zahlen $x$ die Ungleichung \\
 $\sqrt{6x^2-12x+7} \le 3x-2$ erfüllt ist.
\bigskip

\textbf{A2017} \\
Gegeben ist das Polynom $p(x) = x^3-x^2 -2x+8$ mit $x \in \mathbb{R}$. \\
a) Zeigen Sie, dass $p$ die Nullstelle $x = -2$ besitzt. \\
b) Beweisen Sie, dass $p$ keine weitere reelle Nullstelle besitzt. \\
c) Bestimmen Sie alle drei $x$-Werte, für die $p(x)$ den Wert 8 annimmt. \\
d) Bestimmen Sie die Lösungsmenge der Ungleichung $p(x) \le 8, x \in \mathbb{R}$.
\bigskip

\textbf{A2016} \\
a) Skizzieren Sie den Graphen der Funktion $f$ mit 
$f(x) = \vert x + 5 \vert - \vert x+2 \vert$ für $x \in \mathbb{R}$. \\
b) Bestimmen Sie alle reellen Lösungen der Gleichung 
 $\vert x + 5 \vert - \vert x+2 \vert = x + 3$. \\
c) Bestimmen Sie die Lösungsmenge der Ungleichung 
 $\vert x + 5 \vert - \vert x+2 \vert \le x + 3$. 
 \bigskip
 
 \newpage
 \textbf{Sonstige Aufgaben}
 \bigskip
 
 \textbf{A1 Nullstellen} \\
 Berechnen Sie die reellen Nullstellen folgender Polynome ohne Taschenrechner: \\
 a) $p(x) = x^4+2x^3+x^2$ \\
 b) $p(x) = x^2 -2x -15$ 
 \bigskip
 
 \textbf{A2 Polynomdivision} \\
Führen Sie die angegebenen Polynomdivisionen durch. \\
 a) $(2x^3+4x^2-2x-4) : (x-1)$ \\
 b) $(x^3-x^2+3x-3) : (x-2)$
 \bigskip
 
 \textbf{A3 Polynomdivision} \\
 Faktorisieren Sie folgende Polynome in Linearfaktoren: \\
 a) $p(x) = x^3+3x^2-4x-12, x \in \mathbb{R}$ \\
 b) $p(x) = x^3 + x^2 -2x-2,    x \in \mathbb{R}$ \\
 c) $p(x) = x^3 + x^2 -3x+1,    x \in \mathbb{R}$ 
 \bigskip
 
 \textbf{A4 Polynomdivision} \\
Begründen Sie, warum sich das Polynom $p(x) = x^2 +1$ nicht in reelle Linearfaktoren zerlegen lässt.
 \bigskip
 
 \textbf{A5 Ungleichungen} \\
 Bestimmen Sie jeweils die Lösungsmenge der angegebenen Gleichung oder Ungleichung für reelle x.\\
 a) $\vert x- 5 \vert = \vert x \vert +2 $ \\
 b) $(6x-5)(x+1)(x-2) \ge 0$ \\
 c) $\dfrac{x}{x-2} \ge \dfrac{3}{(x-2)^2}$ \\
 d) $\dfrac{2}{x-1} > \dfrac{1}{x}$ \\
 e) $\vert x-2 \vert + \vert 4 - x \vert \le x + 1 $ \\
 f) $\dfrac{x+1}{x-1} > 2$
 
 \newpage
 
 \textbf{A6 Ungleichungen} \\
 Lösen Sie die Ungleichungen und stellen Sie die Lösungsmenge graphisch in einem Koordinatensystem dar. \\
 a) $\vert x \vert + 2 \vert y \vert \ge 4$ \\
 b) $\vert x - 2 \vert + 2\vert y + 1 \vert \ge 4$
 \bigskip
 
 \textbf{A7 Wurzelgleichung}  \\
 Bestimmen Sie die Lösungsmenge der angegebenen Gleichungen. \\
 a) $\sqrt{x+2} + x = 4,  x \in \mathbb{R}$ \\
 b) $\sqrt{x+2} = 10,  x \in \mathbb{R}$ \\
 c) $\sqrt[3]{x-1} + 10 = 12,  x \in \mathbb{R}$ 
 \bigskip
 
  \textbf{A8 Wurzelgleichung} \\
 Bestimmen Sie die Lösungsmenge der angegebenen Wurzelgleichungen. \\
 a) $\sqrt{4x} -\sqrt{2x+7}=1,  x \in \mathbb{R}$ \\
 b) $\sqrt{x+30} = 6 \cdot \sqrt{x-5},  x \in \mathbb{R}$ \\
 c) $\sqrt{x} = \sqrt{x+8} - 2,  x \in \mathbb{R}$ 
 \bigskip
 
\textbf{A9 Wurzelgleichung} \\
Gegeben ist die Gleichung $\sqrt{x-6} + \sqrt{x+2} = 2, x \in \mathbb{R}$ \\
Bestimmen Sie die Lösungsmenge.
\bigskip

\textbf{A10 Wurzelungleichung} \\
Bestimmen Sie jeweils die Lösungsmenge der angegebenen Ungleichungen. \\
a) $\sqrt{x^2+9} + x \le 5,  x \in \mathbb{R}$ \\
b) $\sqrt{x} \cdot \sqrt{x+6} \le 4,  x \in \mathbb{R}$ \\
c) $\sqrt{x+2} + x \le 4,  x \in \mathbb{R}$ 
\bigskip


\end{document}
