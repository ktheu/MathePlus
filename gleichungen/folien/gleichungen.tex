\documentclass[11pt]{beamer}
\usepackage[ngerman]{babel}
\usepackage[utf8]{inputenc}
\usepackage{amsmath}
\usepackage{amssymb}
\usepackage{listings} 
\usepackage{stmaryrd}
\lstset{language=Python, tabsize=4, showstringspaces=false,basicstyle=\footnotesize,mathescape=true} 
\lstset{literate=%
  {Ö}{{\"O}}1
  {Ä}{{\"A}}1
  {Ü}{{\"U}}1
  {ß}{{\ss}}1
  {ü}{{\"u}}1
  {ä}{{\"a}}1
  {ö}{{\"o}}1
}
\usepackage{mathtools}
\usepackage{ulem}
\usepackage{tikz}

\usetheme{Boadilla}
\mode<presentation>{
\useoutertheme[subsection=false]{miniframes}
\useinnertheme{rectangles}
%\usecolortheme{crane}
}
\parskip 10pt

\begin{document}
\title{Vertiefungskurs Mathematik}   
\author{Gleichungen und Ungleichungen} 
\date{}
\frame{\titlepage} 

%---
\begin{frame}[fragile]

Lösungen für Gleichungen oder Ungleichungen können wir auf unterschiedliche Weisen angeben.
\begin{itemize}
\item Aufzählen der Lösungen:  $x_1 = 2, x_2 = -3$ \\
\item Lösungsmenge in Mengenschreibeweise: $\mathbb{L} = \{2; -3\}$ \\
\item Mengenschreibeweise mit einer charakterisierenden Eigenschaft, z.B:
 $\mathbb{L} = \{k \mid k \in \mathbb{Z} \land -2 \le k < 5 \}$ \\
 $\mathbb{L} = \{k \in \mathbb{Z} \mid -2 \le k < 5 \}$ \\
 $\mathbb{L} = \{2k \mid k \in \mathbb{Z} \}$ \\
\item Angabe der Lösungsmenge als Intervall:  $\mathbb{L} =(-1, 2]$ 

\end{itemize}
 
\end{frame}

%---
\begin{frame}[fragile]
\textbf{Polynomgleichungen}

Eine reelle Polynomgleichung ist eine Gleichung, die man auf die Form $f(x) = 0$ mit
einem Polynom $f(x) = a_n x^n + a_{n-1} x^{n-1} + . . . + a_1 x + a_0$ mit reellen Koeffzienten
$a_i \in \mathbb{R}$ für $i = 0, . . . , n$ und $a_n \ne 0$, bringen kann. Dabei heißt $n$ Grad der Gleichung.
Eine reelle Zahl $x$ heißt Lösung der Gleichung,  wenn $x$ eine Nullstelle des Polynoms $f$ ist, also $f(x) = 0$ gilt.

\footnotesize
Beispiele:

\quad $\frac{3}{7}x + 5 = \frac{1}{2}$,   \quad $n = 1$,  lineare Gleichung

\quad $x^2 -5x +2 = 0$,   \quad $n = 2$,  quadratische Gleichung
\end{frame}

%---
\begin{frame}[fragile]
 
Für quadratische Gleichungen $ax^2+bx+c=0$ gibt es manchmal schnellere Lösungswege als die Anwendung der
Mitternachtsformel.

1. Falls $c=0$: Ausklammern und Satz vom Nullprodukt anwenden: 

\footnotesize
Beispiel: $2x^2-3x = 0$. Ausklammern ergibt $x(2x-3)=0$. Mit dem Satz vom Nullprodukt erhalten wir die Lösungen
$x_1 = 0, x_2 = \frac{3}{2}$

\normalsize
2. In der \textit{normierten Form (pq-Form)} ist $a=0$.

Satz von Vieta: Hat die Gleichung  $x^2 +px +q =0$ die Lösungen $x_1$ und $x_2$, so gilt:
$x_1 + x_2 = -p, \quad x_1 \cdot x_2 = q$ 

\footnotesize
Beispiel: $3x^2+3x-18=0$ bringen wir auf die normierte Form $x^2+x-6=0$. Der Satz von Vieta liefert uns $x_1=-3, x_2 = 2$.


\end{frame}


\end{document}