\documentclass[landscape,twocolumn,a4paper]{article}
\usepackage[utf8x]{inputenc}
\usepackage[ngerman]{babel}
\usepackage{listings}
\usepackage{babel}

\usepackage[T1]{fontenc}
\usepackage{booktabs} % schöne Tabellen
\usepackage{graphicx}
\usepackage{csquotes} % Anführungszeichen
\usepackage{paralist} % kompakte Aufzählungen
\usepackage{amsmath,textcomp,tikz} %diverses
\usepackage{eso-pic} % Bilder im Hintergrund
\usepackage{mdframed} % Boxen
\usepackage{multirow}
\usepackage{amssymb}

\usepackage{mathtools}
\usepackage[top=20mm,left=10mm,right=10mm,bottom=10mm]{geometry}
\usepackage{fancyhdr}
\pagestyle{fancy}
\fancyhead[L]{Aufgaben zur Linearen Algebra}
\fancyhead[R]{\thepage}
\fancyfoot{}

\lstset{language=Python, tabsize=4, basicstyle=\footnotesize, showstringspaces=false, mathescape=true}
\lstset{literate=%
  {Ö}{{\"O}}1
  {Ä}{{\"A}}1
  {Ü}{{\"U}}1
  {ß}{{\ss}}1
  {ü}{{\"u}}1
  {ä}{{\"a}}1
  {ö}{{\"o}}1
}
\newcommand{\matrixdd}[9]{\ensuremath{\begin{bmatrix*}[r] #1 & #2 & #3 \\  #4 & #5 & #6\\ #7 & #8 & #9 \\ \end{bmatrix*}}}
\newcommand{\vectord}[3]{\ensuremath{\begin{bmatrix*}[r] #1 \\#2 \\#3 \\ \end{bmatrix*}}} 
\begin{document}
\newcounter{y}
\setcounter {y} {1}
\parindent 0mm
\bigskip


  
\textbf{A\arabic {y}:}   
Schreibe das Gleichungssystem in Matrixform und löse mit einer einfachen Linearkombination der Spalten.
 \begin{align*} 
a.  &  &  b. & & c. \\
 x_{1} + x_{2} - 2 \, x_{3} = 0 & &  2 \, x_{2} + 2 \, x_{3} = 4   & &   3 \, x_{1} + x_{2} = 4   \\
 x_{1} - x_{2} + 2 \, x_{3} = 2  & & 2 \, x_{1} + 2 \, x_{3} = 0  & & x_{1} - x_{2} - 3 \, x_{3} = 3   \\
 3 \, x_{1} + x_{2} = 2 & & -x_{1} + x_{2} + x_{3} = 3  & & -x_{1} + x_{2} + x_{3} = -1 
 \end{align*} 
 

 
\bigskip \stepcounter{y}
 
\textbf{A\arabic {y}:}   
Löse $Ax = b$ mit einer einfachen Linearkombination.
$A =\matrixdd{1}{1}{-2}{1}{-1}{2}{3}{1}{0}$\\

a.  $b = \vectord{0}{2}{2}$ \quad b. $b = \vectord{-1}{1}{1}$ \quad c. $ b = \vectord{-2}{0}{-2}$ 
\bigskip \stepcounter{y}

\textbf{A\arabic {y}:}   
Löse $Ax = b$ mit Elimination und Rücksubstitution. \\

a.  $A =\matrixdd{2}{1}{-2}{-1}{-2}{2}{3}{0}{-1}$ \quad $b = \vectord{-2}{9}{7}$  \quad
b.  $A =\matrixdd{2}{-4}{5}{3}{3}{7}{4}{-2}{3}$ \quad $b = \vectord{-1}{16}{-3}$ 
\bigskip \stepcounter{y}

\textbf{A\arabic {y}:}   
a.
$A =\matrixdd{1}{2}{3}{4}{3}{2}{1}{4}{2}$  \quad
 $B =\begin{bmatrix*}[r] 3 & 1 & 2 & 1\\  2 & 0 & 1 & 0\\ 1 & 1 & 3 & 0\\ \end{bmatrix*}$  \quad
 $C = A \cdot B$  \\ 
 (1) Bestimme den Typ (n x m) von C \\
 (2) Berechne $c_{3,2}$ - Notiere die Rechnung als Skalarprodukt.\\
 (3) Berechne die 1. Spalte von C  - Notiere die Rechnung als Linearkomination geeigneter Vektoren. \\
 (4) Berechne die 2. Zeile von C  - Notiere die Rechnung als Linearkomination geeigneter Vektoren. 

 \vspace{10pt}
 
 b.
$A =\begin{bmatrix*}[r] 1 & 2\\  3 & 4 \\  3 & 2\\ 1 & 4 \\ \end{bmatrix*}$  \quad
 $B =\begin{bmatrix*}[r] 3 & 1 & 2 \\  1 & 2 &  0 \end{bmatrix*}$  \quad
 $C = A \cdot B$   \\
 (1) Bestimme den Typ (n x m) von C \\
 (2) Berechne $c_{3,2}$ - Notiere die Rechnung als Skalarprodukt.\\
 (3) Berechne die 2. Spalte von C  - Notiere die Rechnung als Linearkomination geeigneter Vektoren. \\
 (4) Berechne die 4. Zeile von C  - Notiere die Rechnung als Linearkomination geeigneter Vektoren. 

\bigskip \stepcounter{y}

\newpage
\textbf{A\arabic {y}:}   
Aus Matrix A ist die Stufenform U durch die angegebenen Operationen entstanden. 
Schreibe für jede Operation eine Matrix und formuliere mit diesen Matrizen eine Matrixgleichung die A und U verbindet (die Matrix-Multiplikationen müssen nicht durchgeführt werden) \\
 
a.
 $A = \left[\begin{array}{rrr}
3 & 5 & 9 \\
1 & 2 & 3 \\
-2 & -3 & -7
\end{array}\right] \quad  $  
$U = \left[\begin{array}{rrr}
1 & 2 & 3 \\
0 & -1 & 0 \\
0 & 0 & -1
\end{array}\right]  \quad$
 
 (1) Vertausche Zeile 1 und 2 \\
 (2) Addiere zu Zeile 2 das -3-fache von Zeile 1\\
 (3)  Addiere zu Zeile 3 das 2-fache von Zeile 1 \\
 (4)  Addiere zu Zeile 3 das 1-fache von Zeile 2 
 \vspace{10pt}
 
b.
 $A = \left[\begin{array}{rrr}
2 & 1 & 4 \\
-6 & -5 & -8 \\
8 & 3 & 17
\end{array}\right] \quad  $  
$U = \left[\begin{array}{rrr}
2 & 1 & 4 \\
0 & -1 & 1 \\
0 & 0 & 2
\end{array}\right]  \quad$

 (1) Vertausche Zeile 2 und 3 \\
 (2)  Addiere zu Zeile 2 das -4-fache von Zeile 1 \\
 (3) Addiere zu Zeile 3 das 3-fache von Zeile 1  \\
 (4) Addiere zu Zeile 3 das -2-fache von Zeile 2
 \vspace{10pt}

\bigskip \stepcounter{y}

\textbf{A\arabic {y}:}   
Ermittle die LU-Faktorisierung für Matrix A.

a.  $A =\begin{bmatrix*}[r] 1 & 3 & 2\\  -2 & -3 & -5\\ 0 & 9 & 1\\ \end{bmatrix*}$  \quad 
b.  $A =\begin{bmatrix*}[r] 1 & -1 & 1\\  -1 & 0 & 0\\ 1 &-3 & 5\\ \end{bmatrix*}$  \quad 

\bigskip \stepcounter{y}

\textbf{A\arabic {y}:}   
Ermittle mit dem Gauß-Jordan-Algorithmus die Inverse der Matrix  A.

a.  $A =\begin{bmatrix*}[r] 1 & 3 & 1\\  0 & 1 & 2\\ 1 & 3 & 2\\ \end{bmatrix*}$  \quad 
b.  $A =\begin{bmatrix*}[r] 3 & 2 & 2\\  -3 & -4 & -3\\ -4 & -3 & -3\\ \end{bmatrix*}$  \quad \\
\bigskip \stepcounter{y}

\textbf{A\arabic {y}:}   
Gegeben sei eine 5x5  Matrix A. Die Matrix B gehe aus A durch die angegebenen Operationen hervor. 
Gib eine Matrix P an, die die angegebenen Operationen durchführt und formuliere eine Gleichung mit A,B und P.


a. Tausch der 1. mit der 4. Zeile, dann Tausch der 5. mit der 2. Zeile, \\
dann Tausch der 2. mit der 3.Zeile. \\
b. Tausch der 1. mit der 3. Spalte, dann Tausch der 5. mit der 4. Spalte.

\bigskip \stepcounter{y}

\textbf{A\arabic {y}:}   
Finde alle Lösungen für Ax = b.

a.   $A =\begin{bmatrix*}[r] 1 & 3 & 0 & 2\\  0 & 0 & 1 & 4\\ 1 & 3 & 1 & 6\\ \end{bmatrix*}$  \quad 
      $b =\begin{bmatrix*}[r] 1 \\  6\\ 7\\ \end{bmatrix*}$  \quad 
b.   $A =\begin{bmatrix*}[r] 1 & 2 & 3 & 5\\  2 & 4 & 8 & 12\\ 3 & 6 & 7 & 13\\ \end{bmatrix*}$  \quad 
      $b =\begin{bmatrix*}[r] 0 \\  6\\ -6\\ \end{bmatrix*}$  \quad 
\bigskip \stepcounter{y}

\textbf{A\arabic {y}:}   
Lies aus der Matrix A und ihrer rref-Form folgende Angaben ab: Anzahl pivots und Anzahl freier Variablen bei der Elimination, 
Rang(A), Dimension  Bild(A), Dimension Kern(A), eine Basis von Bild(A), eine Basis von Kern(A).

a.   $A =\begin{bmatrix*}[r] 2 & 3 & -4 & 2\\  1 & 0 & 1 & 1\\ 2 & -1 & 4 & -1\\ -1 & -2 & 3 & 2\end{bmatrix*}$  \quad 
      $\text{rref}(A) =\begin{bmatrix*}[r] 1 & 0 & 1 & 0\\  0 & 1 & -2 & 0\\ 0 & 0 & 0 & 1\\ 0 & 0 & 0 & 0\end{bmatrix*}$ 
\\  
\.
\\   
b.   $A =\begin{bmatrix*}[r] 3 & 1 & 4 & 3 & -4\\  1 & 1 & 0 & -1 & 2\\ 2 & -1 & 6 & 7 & -11\\ -1 & -2 & -6 & -8 & 13\end{bmatrix*}$  \quad 
      $\text{rref}(A) =\begin{bmatrix*}[r] 1 & 0 & 2 & 2 & -3\\  0 & 1 & -2 & -3 & 5\\ 0 & 0 & 0 & 0 & 0\\ 0 & 0 & 0 & 0 & 0\end{bmatrix*}$ 
\bigskip \stepcounter{y}
\end{document}
